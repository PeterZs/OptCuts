% !TeX root = OptCuts.tex

\section{Conclusions}
\label{sec:conclusion}

We propose OptCuts, a novel algorithm for mesh parameterization that minimizes seam length while satisfying a target distortion bound. OptCuts jointly optimizes over the space of discrete topological changes and continuous embedding parameters. In all examples tested, across all deformation bounds applied, OptCuts successfully finds a local solution with low seam length measure, satisfying the applied distortion bound while reaching either an exact or cyclical stationary point of our optimization. 

OptCuts is versatile. In addition to creating bounded-distortion, locally injective maps with low seam lengths, we have additionally demonstrated how OptCuts can generate globally bijective maps and enforce user constraints on seam placement when desired. Likewise, along with generating high-quality maps from scratch we have also shown how OptCuts can be applied to polish pre-existing UV-maps from other tools and pipelines while ensuring bounds on distortion are preserved or even improved. Comparing OptCuts to state-of-the art methods and commercial tools across a wide range of examples we find that in the vast majority of cases OptCuts yields shorter seams while achieving the same or lower distortion. 

\subsection{Limitations and Future Work}

Our focus has been on automatically generating high-quality parameterizations that satisfy user-provided distortion bounds while minimizing seam length. Numerically, in each inner iteration OptCuts is explicitly designed to decrease the Lagrangian. As a result we observe that OptCuts so far efficiently converges in all examples to either a fixed or cyclical stationary point satisfying a rough numerical optimality without any need to apply standard heuristic upper bounds on iteration counts. While OptCuts works well in practice, but we are unable to prove its optmality. Further analysis and understanding of both cyclical points and the overall convergence behavior are important future work here. Likewise, as discussed earlier we so far find our current set of topology operations and initialization points to be effective and efficient; however, it remains an interesting and open avenue of future work to seek improved mesh operation subsets and initializers for OptCuts to improve the convergence and quality of the obtained maps even further. In terms of distortion measures we have so far focused solely on the symmetric Dirichlet energy, but our formulation is largely agnostic to choice of distortion measure.  It would also be interesting to swap in alternate measures for e.g.\ conformal distortion. In terms of efficiency, OptCuts could support a range 
of additional parallelism, for example, many topological operations could be executed simultaneously in partitioned regions of the mesh, while linear system solves seem reasonable to decompose as well. Finally, in many cases we envision incorporating important additional priors to, for example, favor seamless parameterization, seam smoothness, and the creation of charts that efficiently use texture space. 


%\paragraph{Limitations and Future Works}
%Our formulation for shape parameterization as a constrained optimization with discrete-continuous search opens novel research challenges. First, it would be interesting to further investigate the space of energy wells, and gain better insights on how initial conditions affect the quality of the final solution. Second, we would like to extend the running time of our method by leveraging some parallelism, for example, some topological searchers can be executed at the same time on distant mesh regions. Third, it would be useful to incorporate additional priors into our framework to favor seamless parameterization, seam smoothness, and creation of charts that efficiently use texture space. 
%
%\vova{Minchen has a great list of other ideas for future work (see comments), some points can be added back}

%Our method does not provide globally optimal solutions, the results are still locally optimal, but w.r.t. both seam placement and distortion, which is better than previous 2-pass methods that breaks the correlation between seam placement and distortion.
%
%\begin{itemize}
%\item take advantage of basic SIMD type parallelism for improving results quality by directly evaluating $f_v$ for neighbors and track multiple branches, very useful for practical implementations
%\item multiple fracture
%\item control seam smoothness
%\item seamless parameterization
%\item conformal parameterization
%\item given a symmetric shape, whether symmetrically triangulated or not, generate symmetric UV map
%\item if slightly modifying the triangulation is allowed, we could also create fractures in the interior of an element and locally remesh the stencil, which makes our method more triangulation and resolution invariance
%\item start with input surface and solve in 3D by reducing the curvature. In this way, the need for locally injective initial embedding in parameterization problems could be eliminated, and the result is only "biased" by its 3D shape, which is the most reasonable bias
%\item dynamic impulse for bijectivity parameterization within our framework
%\end{itemize}
%
%In summary, we proposed a framework that tackles the problems related to both mesh topology and an energy function defined on the surface geometry, thus we expect our framework to be applied on more discrete-continuous geometry processing problems.
%
%\paragraph{Limitations and Future Works}

%\section*{Acknowledgements} 