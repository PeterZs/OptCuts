% !TeX root = OptCuts.tex

\section{Conclusions}
\label{sec:conclusion}

\paragraph{Limitations and Future Works}
Our method does not provide globally optimal solutions, the results are still locally optimal, but w.r.t. both seam placement and distortion, which is better than previous 2-pass methods that breaks the correlation between seam placement and distortion.

\begin{itemize}
\item take advantage of basic SIMD type parallelism for improving results quality by directly evaluating $f_v$ for neighbors and track multiple branches, very useful for practical implementations
\item multiple fracture
\item control seam smoothness
\item seamless parameterization
\item given a symmetric shape, whether symmetrically triangulated or not, generate symmetric UV map
\item if slightly modifying the triangulation is allowed, we could also create fractures in the interior of an element and locally remesh the stencil, which makes our method more triangulation and resolution invariance
\item start with input surface and solve in 3D by reducing the curvature. In this way, the need for locally injective initial embedding in parameterization problems could be eliminated, and the result is only "biased" by its 3D shape, which is the most reasonable bias
\item dynamic impulse for bijectivity parameterization within our framework
\end{itemize}

In summary, we proposed a framework that tackles the problems related to both mesh topology and an energy function defined on the surface geometry, thus we expect our framework to be applied on more discrete-continuous geometry processing problems.

\section*{Acknowledgements} 