% !TeX root = OptCuts.tex

\begin{algorithm}[h]
\SetAlgoLined
\KwData{$M$, $T^{k,i-1}$, $U^{k,i-1}$, $\delta^{k,i-1}$, $\lambda^{k+1}$}
\KwResult{$T^{k,i}$, $U_a^{k,i-1}$}
$\hat{\mathcal{E}}^{k,i-1}_T \leftarrow$ operationFiltering($M$, $T^{k,i-1}$, $U^{k,i-1}$); // Section~\ref{sec:operationFiltering}\\
\For{each $e^{(k,i-1),j}_{T}$ in $\hat{\mathcal{E}}^{k,i-1}_T$}{
  $\hat{f}_e(e^{(k,i-1),j}_{T}) \leftarrow \Big(E^j_s + \lambda^{k+1} \min_{U^{(k,i-1),j}} E_d(T^j,U)\Big) - L(T^{k,i-1},U^{k,i-1},\lambda^{k+1})$\;
}
$U_a^{k,i-1} \leftarrow U^{k,i-1}$\;
\If{$\min_{e^{(k,i-1),j}_{T} \in \hat{\mathcal{E}}^{k,i-1}_T} \hat{f}_e(e^{(k,i-1),j}_{T}) \leq \delta^{k,i-1}$}
{
  $T^{k,i} \leftarrow T^j$, $U_a^{(k,i-1),j} \leftarrow \argmin_{U^{(k,i-1),j}} E_d(T^j,U)$\;
}
\caption{Topology Descent Step $(k+1,i)$}
\label{alg:topologyStep}
\end{algorithm}

\section{Coupled Discrete-Continuous Descent}
%\section{Coupled Topology and Distortion Descent}
\label{sec:topologySearch}
\danny{Did a pretty much full reworking of this section and the last two sections as well, if we're happy with this then we can clean up the pseudo code to make it consistent. Haven't touched the pseudocode yet.}
To perform our primal update we seek to minimize our Lagrangian
%, composed of the weighted sum of seam-length, $E_s$ and distortion, $E_d$ jointly 
over both continuous changes in vertex position \emph{and} discrete changes in topology. 
%
In order to optimize over topology we could potentially perform exhaustive search over the graph of all possible mesh changes. However, this clearly becomes impractical for any practical size mesh. 
Instead, we begin by recalling the standard process of optimizing a smooth energy solely over vertex DoFs. In such cases, each iterate generally forms a localized approximation of the minimized energy and then uses it to propose a candidate direction for decrease. Then, as finding such a direction is expensive, we apply a search along the proposed direction to ensure we will gain a significant decrease in energy (and not increase) along it. 

Here we extend this process to include explorations of discrete variations in topology. In analogy to seeking a continuous search direction, at the start of each new primal solve we will build many localized energy models to search in parallel for a likely candidate mesh operation to repeatedly propagate topology change, i.e., cutting or merging, over our UV mesh. Each inner iterate of the primal solve will successively apply this operation in combination with standard smooth descent to explore discrete-continuous descent until no further progress is made and so a primal update is found.    

\subsection{Energy Model}

At each outer iterate $k$ the Lagrangian's energy for any proposed topology $T^i$ is  
\begin{align}
\ell(T^i) = \min_{U} L(T^i, U) = E_s(T^i) + \lambda^k \min_{U} E_d(T^i, U).
\end{align}
Then, for any valid topology changing operation, $o^{i,j}:T^i \rightarrow T^j$, the resultant change in energy is $\Delta \ell(o^{i,j}) = \ell(T^j) - \ell(T^i)$.

Starting from known $(T^i, U^i)$ we approximate change in the Lagrangian by restricting the distortion update to the locally changed vertex stencil $U^{i,j}$ of the applied topological operation $o^{i,j}$ (see below for the stencils). 
We do this by holding all other remaining vertices $U_s$, shared in common with $T^i$, fixed in the mesh
to the same positions previously held in $U^i$, so that we have $U^j = (U^{i,j}, U_s)$.  Our approximate change in energy model is then 
%\danny{TODO: Still need some syntatic sugar to make this decomposition of fixed and free new vertices clean.}
\begin{align} 
d(o^{i,j}) = E_s(T^i) + \lambda^k \min_{U^{i,j}} E_d \big( T^i, (U^{i,j}, U_s) \big) - L(T^i,U^i,\lambda^k).
\end{align}

\subsection{Local Topological Operations}

We consider descent with topology changes propagated by three mesh operations.

\paragraph{Boundary Vertex Split}
Boundary vertices can be split along any interior incident edges. Each such candidate split generates two duplicate vertices forming the stencil to compute $d$.
When splitting a boundary vertex along an edge connecting to another boundary vertex we either remove a hole or else produce a new chart in our UV map. For the latter case, we generate four duplicate vertices forming the stencil to compute $d$.
%\minchen{[TODO] 4-vertex merge to support joining two charts together, triangle moving operations?}

\paragraph{Interior Vertex Split}
Interior vertices can be split along any pair of incident edges. Each such candidate split generates two duplicate vertices forming the stencil to compute $d$. 
%We ignore potential interior splits when this would be connected to an existing seam, as this is overloaded with propagating the boundary vertex split operation.

\paragraph{Corner Merge}
Corners are formed by three UV vertices corresponding to the tail edge of a cut seam on the input surface. Merging the end vertices generates a single new vertex forming the stencil to compute $d$. Merging requires extra care here. Unlike vertex splitting, an initial location for the newly merged vertex must be selected. Naive merging can violate local-injectivity and so prevent progress if we are working with barrier-type energies like symmetric Dirichlet. We initialize a merged vertex to the average of its parent vertices. If inversion is detected, we then apply Agmon's relaxation~\shortcite{Agmon1954Relaxation} to iteratively project to an inversion-free position. On rare occasions this will not suffice and so we remove the proposed operation from our queue; see below.
%However, cases are still there when only moving the merged vertex is not enough to obtain an inversion free initial local stencil. In these situations, we will just abandon the candidate. This rarely happens in practice and does not affect our result.\justin{didn't follow the last two sentences, maybe needs a small figure}

\subsection{Topology Search Candidates}
In analogy to a continuous search direction, at the start of each new primal solve we search for a candidate mesh operation to propagate descent with. 
%
To select the candidate mesh operation we first consider all $m$ boundary vertices in the current topology $T^k$. \danny{Minchen: confirm this detail:} For each such vertex we compute the standard deviation over all the distortion energy gradients acting on the vertex contributed from incident elements. From the set of $m^{0.8}$ boundary vertices with largest deviation we then build a set of candidate mesh operations $O^k$ formed by boundary vertex splits and corner merges (when valid) on those vertices. We then find a minimizer
\begin{align}
\label{eq:minO}
o^{i,j} = \argmin_{o \in O^k} d(o).
\end{align}
Note that with local support all queried $d$ in this minimization can be evaluated in parallel. If the minimizing operation promises descent, i.e., $d(o^{i,j}) < 0$, we seed our candidate mesh operation with $s^k = o^{i,j}$ to our descent process with this operation. Otherwise we go on to find the $(n-m)^{0.8}$ interior vertices with largest deviation and rebuild $O^k$ as the set of all interior vertex splits on those vertices. A $d$ minimizing interior split operation $o^{i,j} \in O^i$, again using(\ref{eq:minO}), is then set as the seed to $s^k = o^{i,j}$ for our descent process.

\subsection{Iterated Search, Propagation and Descent}

Once we have computed our candidate search operation $s^k$, we begin an inner loop iteration, indexed by superscript $i$ (recalling outer iterates are indexed by $k$), to solve the primal variable update. At the start of this process we initialize our desired minimum amount of estimated decrease to $\delta^0 = 0$; this will be updated at each successive iterate. Then, each inner iteration $i$ proceeds by first propagating topology change followed by smooth coordinate update. 

Each topology propagation step first generates a set of all possible mesh operations $\mathcal{E}(s^k)$ that could continue the seed operation $s^k$, e.g. all edges that could continue to propagate a boundary splitting cut from the current tail. See Figure\ \ref{}  \danny{Minchen please add the figure here.}.
As in seeking our seed operation we once agin find the $d$ minimizing operation from this small set of candidate operations
\begin{align}
\label{eq:minE}
o^{i,j} = \argmin_{o \in \mathcal{E}(s^k)} d(o).
\end{align}
If this minimizer provides sufficient estimated decrease, so that $d(o^{i,j}) \leq \delta^{i-1}$, we accept the topology update setting $T^i \leftarrow T^j$; otherwise, we keep topology unchanged with $T^i \leftarrow T^{i-1}.$ 

The following coordinate update then simply applies a single step of projected Newton descent with line search to gain the updated vertex coordinates $U^i$; see Section~\ref{} for details. We then ask for the next topology update to gain similar or greater magnitude decrease by setting $\delta^i = E_d(U^i) - E_d(U^{i-1})$.

This process terminates at iteration $i+1$ when smooth iterations have converged (either by gradient or relative error norm) \emph{and} the propagation of the seed mesh operation produces no further descent. We then set $(T^k,U^k) \leftarrow (T^{i+1},U^{i+1})$ and begin the next outer iterate $k+1$ with the dual variable update. 


