% !TeX root = OptCuts.tex

\section{Related Works}

related methods:\\
AutoCuts~\cite{Poranne2017Autocuts}\\
Seamster~\cite{Sheffer2002Seamster}\\
geometry images~\cite{Gu2002Geometry}\\
Multi-chart geometry images~\cite{Snyder2003Multi}\\
D-Chart~\cite{Julius2005D}\\
Boundary First Flattening~\cite{Sawhney:2017}\\
SeamCut~\cite{Lucquin:2017}\\
Bijective parameterization with free boundaries~\cite{Smith2015Bijective}\\
MIPS~\cite{Hormann2000MIPS}\\
ABF++~\cite{Sheffer2005ABFPP}
global parameterization methods?

%Seams, due to its discontinuous property, is not intuitive to be considered in traditional distortion minimization frameworks.
Due to discontinuities that occur when seams are introduced or removed, it is not intuitive to consider optimization of seam topology in the context of traditional frameworks for minimizing distortion during mesh parameterization.
%
\justin{couldn't follow this sentence (what does ``efficient'' or ``sparse'' mean in this context and what does it have to do with L2?):}\minchen{"efficient" and "sparse" means the number of seam edges should be in $O(\sqrt{numOfEdges})$. If one tries to model seam as a continuous energy (e.g.\ \cite{Poranne2017Autocuts}), residuals will be distributed evenly if the objective function is L2-type. To obtain sparsity structure in the result, less smooth energy such as L1-type energy is needed, which is similar to feature selection in statistical learning.}
Moreover, for seams to be efficient, it needs to be sparse, which is another challenge for optimizing it with L2-type distortion energies.


The recent AutoCuts~\cite{Poranne2017Autocuts} algorithm measures seam quality using an energy with discontinuities when triangles are glue together or disconnected.  Their procedure progressively builds up a parameterization starting from triangle soup, jointly improving topology and distortion via homotopy optimization. 
%
%We observed that i
While their method is among the first to optimize parameterization topology and geometry simultaneously, 
initially placing seams on all the edges introduces unneeded degrees of freedom and unnecessary computational expense:  Most of the triangles remain attached to their neighbors after their optimization procedure converges. Also, since their seam placement highly depends on the homotopy path, AutoCuts relies on user guidance to obtain good results, e.g.\ for parameter tuning, cut suggestion, and patch movement. %\danny{Here or elsewhere we should also add the observation that a full triangle soup initializer requires an awful lot of extra (and generally unnecessary) work to glue everything back together...}%\justin{how's the above?}

Our framework is different from well-known seam cutting algorithms like Geometry Images~\cite{Gu2002Geometry} and Seamster~\cite{Sheffer2002Seamster}, in which the core idea is to locate points of maximal currently predicted distortion and to add cut paths toward them. These heuristics do not perform well if no such obvious points exist, e.g.\ once distortion is distributed near-evenly across many surface points. Our framework in contrast searches for minimal cut elongation or shrinking steps that reduce a joint objective, and thus we expect it to be more efficient in such settings (Figure~\ref{cases where there are not many obvious extremal points}).

Although OptCuts does not require user assistance, it still allows users to communicate preferences on regional seam placement through edge weight painting (Figure~\ref{fig:edge_weight_painting}).
In addition, it can work with ``bespoke'' distortion energies when necessary. %our seams are optimal for the distortion energy used. 
For example, it creates different set of seams that benefit conformality if the objective function penalizes conformal distortion (Figure~\ref{results of our method with conformal distortion energy}).