% !TeX root = OptCuts.tex

\section{Related Works}

related methods:\\
AutoCuts~\cite{Poranne2017Autocuts}\\
Seamster~\cite{Sheffer2002Seamster}\\
geometry images~\cite{Gu2002Geometry}\\
Multi-chart geometry images~\cite{Snyder2003Multi}\\
D-Chart~\cite{Julius2005D}\\
Boundary First Flattening~\cite{Sawhney:2017}\\
SeamCut~\cite{Lucquin:2017}\\
Bijective parameterization with free boundaries~\cite{Smith2015Bijective}\\
MIPS~\cite{Hormann2000MIPS}\\
ABF++~\cite{Sheffer2005ABFPP}
global parameterization methods?

Seams, due to its discontinuous property, is not intuitive to be considered in traditional distortion minimization frameworks. Moreover, in order for seams to be efficient, it needs to be sparse, which is another challenge for optimizing it with L2-type distortion energies. The recently published AutoCuts~\cite{Poranne2017Autocuts} model seam as a discontinuous energy using triangle soup data structure and jointly optimize it with distortion via homotopy optimization. We observed that initially placing seams on all the edges introduces multiple times of redundant degree of freedoms since during their solving process, most of the triangles keep the relative position to their neighbors. Besides, since the placement of seams highly depends on the homotopy path, AutoCuts requires a certain amount of user guidance, e.g. parameter tuning, cut suggestion, patch movement, in order to obtain good results. \danny{Here or elsewhere we should also add the observation that a full triangle soup initializer requires an awful lot of extra (and generally unnecessary) work to glue everything back together...}

Our framework is different from traditional seam cutting algorithms such as Geometry Image~\cite{Gu2002Geometry} and Seamster~\cite{Sheffer2002Seamster}, of which the core idea is to locate points of maximal currently predicted distortion and to add paths toward them. They do not perform well if no such obvious points exist, e.g. once distortion is distributed near-evenly across many surface points. Our framework in contrast searches for minimal cut elongation or shrinking steps that reduce the joint objective, thus we expect it to be more efficient in such settings (Figure~\ref{cases where there are not many obvious extremal points}).

Although OptCuts doesn't need any user assistance, it still allows users to communicate preferences on regional seam placement through edge weight painting (Figure~\ref{fig:edge_weight_painting}).
In addition, our seams are optimal for the distortion energy used. For example, it creates different set of seams that benefit conformality more if conformal energy is used (Figure~\ref{results of our method with conformal distortion energy}).