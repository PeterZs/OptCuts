% !TeX root = OptCuts.tex

\section{Joint Discrete-Continuous Search}

To start with, we use symmetric Dirichlet energy~\cite{Smith2015Bijective} as our distortion energy, normalized by surface area:
\[ E_d = E_{SD} = \frac{1}{\sum_{t\in\mathcal{F}} |A_t|} \sum_{t\in\mathcal{F}} |A_t|(\sigma_{t,1}^2 + \sigma_{t,2}^2 + \sigma_{t,1}^{-2} + \sigma_{t,2}^{-2}) \]
where $\mathcal{F}$ is the set of all triangles, $|A_t|$ is the area of triangle $t$ on the input surface, and $\sigma_{t,i}$ is the $i$-th singular value of the deformation gradient of triangle $t$.

For seam energy, we simply use a normalized seam length:
\[ E_s = E_{SL} = \frac{1}{\sqrt{(\sum_{t\in\mathcal{F}} |A_t|)/\pi}} \sum_{i \in \mathcal{S}} |e_i| \]
where $\mathcal{S}$ is the set of all seam edges on the input surface, $|e_i|$ is the original length of edge $i$.

With the energies normalized, our $E_w$ is invariant of coordinate scale and resolution for meshes with the same shape.

However, seam sparsity usually leads to discontinuous energies w.r.t. UV coordinates $U \in \mathcal{R}^{2n_v}$, ($n_v$ is the number of vertices on the input mesh), which is non-trivial to be considered into existing distortion minimization routines. Instead of progressively approximating seam sparsity energy with a continuous counterpart applying homotopy optimization method as~\cite{Poranne2017Autocuts}, we handle this discrete energy in a combinatoric way - searching in the topological space.

\subsection{Formulation}

This topological space is a directed graph $G_T$ with its vertices $v_T \in V_T$ being all possible UV topologies of a given 3D surface, and its edges $e_T \in E_T$ are the basic topological operations conducted on a mesh such as vertex split, edge merge, etc, that can transform one UV topology to a nearby topology.

Now, if we consider both distortion and seam in one objective $E_w$, we can define the value $f_v$ of vertex $v_{T,i}$ as 
\[ f_v(v_{T,i}) = \min_{U_i} E_w \]
and the weights $f_w$ of edge $e_{T,m}$ from $v_{T,i}$ to $v_{T,j}$ could just be defined as 
\[ f_w(e_{T,m}) = f_v(v_{T,j}) - f_v(v_{T,i}) \]
Thus our problem could be written as
\[ \min_{U, v_T} E_w \]
which could be stated as to search for a $v_{T,i}$ on $G_T$ where all edges connected to it satisfies $f_w \geq 0$. 

However, computing $f_v$ for one UV topology requires a whole continuous optimization process, and even the number of neighbors of one UV topology is in the scale of $n_v$. Consequently, we construct a single search path on $G_T$ by progressively introducing or removing seams on the UV map, and we only estimate $f_w$ on a local stencil of $U$ for a filtered set of neighbors on $G_T$ so that the whole process of continuous optimization is only conducted while necessary.

\subsection{Method}

We minimize $E_w$ by iteratively alternate between continuous optimization (in descent steps) and discrete optimization (in topology steps):
\begin{itemize}
\item In descent steps, we compute $\min_{U_i} E_{SD}$ given $v_{T,i}$ via projected Newton method~\cite{Teran2005Robust} so that we obtain
\[ f_v(v_{T,i}) = E_{se,i} + \min_{U_i} E_{SD} \]
\item In topology steps, we estimate $f_v(v_{T,j})$ for a filtered set of neighbors of $v_{T,i}$ on a local stencil of $U$ as $\hat{f}_v$ and move onto the neighbor $v_{T,i+1}$ with smallest $\hat{f}_v$.
\end{itemize}
If after a descent step, $f_v(v_{T,i}) \geq f_v(v_{T,i-1})$ is detected, we stop the process by rolling back to $v_{T,i-1}$, which is the stationary of $E_w$ w.r.t. both UV topology (in an approximation sense) and coordinates that we are searching for.

\subsection{Convergence}

As our method is defined to guarantee convergence, we now analyze the convergence rate. First, it's easy to see that $E_w$ is monotonically decreasing looking at each end of descent steps. Now we look at descent step $i$ and $i+1$, from $E^i_w \geq E^{i+1}_w$ we have
\[ E^i_{SD} - E^{i+1}_{SD} \geq \frac{\lambda}{1-\lambda} (E^{i+1}_{se} - E^i_{se}) \geq \frac{\lambda}{1-\lambda} \frac{1}{\sqrt{(\sum_t |A_t|)/\pi}} |e|_{min} \]
if we now only consider splitting operations that keep increasing $E_{se}$. It's obvious that $E_{SD}$'s theoretical lower bound is defined to be $4$, so we have
\[ n_{alter} \leq \frac{(1-\lambda)\sqrt{(\sum_t |A_t|)/\pi}}{\lambda|e|_{min}} (E^0_{SD} - 4) \]
The most important hint we can read from this is, to accelerate convergence, we can move through multiple vertices on $G_T$ in each topology step to increase $E^{i+1}_{se} - E^i_{se}$. \minchen{[TODO] Consequently, we build an anologous line search method and allow multiple fracture initiation to be appropriately agressive when searching in the topological space and ensure that we won't fall into bad locally optimal UV topologies.}

\minchen{[TODO] update the proof with merge operations}

\section{Descent Steps for Continuous Optimization}

\subsection{Newton-type Iterations}

for each descent step inner iteration $j$:

compute $E_{SD}$ Hessian proxy $P^j$ using projected Newton;

compute $E_{SD}$ gradient $g^j$;

solve for search direction $p^j$ ($P^j p^j = -g^j$) using PARDISO symmetric indefinite solver;

compute initial step size $\alpha^j_0$ by avoiding element inversion;

backtracking line search with Armijo rule;

update $U^{j+1} = U^j + \alpha^j p^j$;

record energy decrease $(1-\lambda_t)\Delta E_{SD}^j$;

\subsection{Potential Accelerations for Practical Use}

Since our topological operations only change the mesh locally both on connectivity and coordinates, we could also update the Hessian or the decomposition locally after topology changes to save time. Besides, it's also interesting to try other Hessian approximation methods like L-BFGS or Majorization to explore further acceleration by finding a balance between computational cost and convergence rate.

For convergence tolerance of descent steps, $||\nabla E_{SD}||^2 \leq 10^{-6}$ (note that our energy is normalized) works generally well for all input models judging from the initiated fracture in the following topology step. In fact more inexact solve performs well on most of the models with even $||\nabla E_{SD}||^2 \leq 10^{-4}$, but some may result even better with $||\nabla E_{SD}||^2 \leq 10^{-8}$. Since we are conducting non-convex optimization, $||\nabla E_{SD}||^2$ is not always decreasing, which is also why we don't use Wolfe conditions for line search. The argument here for tolerance issue is that, it depends on whether we are truly in the infinitesimal region of a stationary. Some configuration with $||\nabla E_{SD}||^2 \leq 10^{-6}$ may still not inside the infinitesimal region of a stationary, where if optimization goes on, the $||\nabla E_{SD}||^2$ will go up and then fall down again to a real stationary, which is understandable in non-convex optimization.

\section{Topology Steps for Discrete Optimization}

\subsection{Evaluating Topological Operations via Optimization on Local Stencils}

Candidate Filtering:
for each vertex
  compute divergence of local gradients
independently picking $\sqrt{n_{v,b}^i}$ boundary vertices and $\sqrt{n_{v,i}^i}$ interior vertices with largest divergence as candidates

Local Evaluation:
for each candidate vertex
  if on boundary
    for each interior incident edge
      split and compute $\Delta E_{SD,l}$ locally
      compute $\Delta E_{w,l} = (1 - \lambda_t) \Delta E_{SD,l} + \lambda_t \Delta E_{se}$
  else
    for each pair of incident edges forming a smooth path
      split and compute $\Delta E_{SD,l}$ locally
      compute $\Delta E_{w,l} = 0.5((1 - \lambda_t) \Delta E_{SD,l} + \lambda_t \Delta E_{se})$

split the vertex with largest $|\Delta E_{w,l}|$
turn on fracture propagation

\textcolor{red}{try larger stencils}

\textcolor{red}{enable merge operation}

\textcolor{red}{\subsection{Line Search in Topological Space}}

Current Fracture Propagation:
if fracture propagation is on
  for each fracture tail vertex $k$
    for each interior incident edge of $k$
      split and compute $\Delta E_{SD, l}$ locally
      compute $\Delta E_{w,l} = (1 - \lambda_t) \Delta E_{SD,l} + \lambda_t \Delta E_{se}$
  if the largest $|\Delta E_{w,l}|$ is larger than $|(1-\lambda_t)\Delta E_{SD}^j|$
    propagate fracture by splitting the vertex
  else
    turn off fracture propagation for the rest of the current descent step