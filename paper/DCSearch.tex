% !TeX root = OptCuts.tex

%\section{Joint Discrete-Continuous Search}
% \section{Descent Steps}%Jointly Discrete and Continuous Search}%<--- sounds weird
% \label{sec:DCSearch}

\section{Descent Steps for Continuous Optimization}

\subsection{Newton-type Iterations}

for each descent step inner iteration $j$:

compute $E_{SD}$ Hessian proxy $P^j$ using projected Newton;

compute $E_{SD}$ gradient $g^j$;

solve for search direction $p^j$ ($P^j p^j = -g^j$) using PARDISO symmetric indefinite solver;

compute initial step size $\alpha^j_0$ by avoiding element inversion;

backtracking line search with Armijo rule;

update $U^{j+1} = U^j + \alpha^j p^j$;

record energy decrease $(1-\lambda_t)\Delta E_{SD}^j$;

\subsection{Potential Accelerations for Practical Use}

Since our topological operations only change the mesh locally both on connectivity and coordinates, we could also update the Hessian or the decomposition locally after topology changes to save time. Besides, it's also interesting to try other Hessian approximation methods like L-BFGS or Majorization to explore further acceleration by finding a balance between computational cost and convergence rate.

For convergence tolerance of descent steps, $||\nabla E_{SD}||^2 \leq 10^{-6}$ (note that our energy is normalized) works generally well for all input models judging from the initiated fracture in the following topology step. In fact more inexact solve performs well on most of the models with even $||\nabla E_{SD}||^2 \leq 10^{-4}$, but some may result even better with $||\nabla E_{SD}||^2 \leq 10^{-8}$. Since we are conducting non-convex optimization, $||\nabla E_{SD}||^2$ is not always decreasing, which is also why we don't use Wolfe conditions for line search. The argument here for tolerance issue is that, it depends on whether we are truly in the infinitesimal region of a stationary. Some configuration with $||\nabla E_{SD}||^2 \leq 10^{-6}$ may still not inside the infinitesimal region of a stationary, where if optimization goes on, the $||\nabla E_{SD}||^2$ will go up and then fall down again to a real stationary, which is understandable in non-convex optimization.

\section{Topology Steps for Discrete Optimization}

\subsection{Evaluating Topological Operations via Optimization on Local Stencils}

Candidate Filtering:
for each vertex
  compute divergence of local gradients
independently picking $\sqrt{n_{v,b}^i}$ boundary vertices and $\sqrt{n_{v,i}^i}$ interior vertices with largest divergence as candidates

Local Evaluation:
for each candidate vertex
  if on boundary
    for each interior incident edge
      split and compute $\Delta E_{SD,l}$ locally
      compute $\Delta E_{w,l} = (1 - \lambda_t) \Delta E_{SD,l} + \lambda_t \Delta E_{se}$
  else
    for each pair of incident edges forming a smooth path
      split and compute $\Delta E_{SD,l}$ locally
      compute $\Delta E_{w,l} = 0.5((1 - \lambda_t) \Delta E_{SD,l} + \lambda_t \Delta E_{se})$

split the vertex with largest $|\Delta E_{w,l}|$
turn on fracture propagation

\textcolor{red}{try larger stencils}

\textcolor{red}{enable merge operation}

\textcolor{red}{\subsection{Line Search in Topological Space}}

Current Fracture Propagation:
if fracture propagation is on
  for each fracture tail vertex $k$
    for each interior incident edge of $k$
      split and compute $\Delta E_{SD, l}$ locally
      compute $\Delta E_{w,l} = (1 - \lambda_t) \Delta E_{SD,l} + \lambda_t \Delta E_{se}$
  if the largest $|\Delta E_{w,l}|$ is larger than $|(1-\lambda_t)\Delta E_{SD}^j|$
    propagate fracture by splitting the vertex
  else
    turn off fracture propagation for the rest of the current descent step

\subsection{Operation Filtering}
\label{sec:operationFiltering}
As the number of vertex that could be splitted is in the scale of $n_p$, and initiating a cut in near-isometric regions will not help improve the UV map much, we filter the vertices to be considered in a split operation by computing the standard deviation of the local individual energy gradients on a vertex and only considering the top $n_p^{0.6}$ vertices with large deviation. \minchen{[TODO] why not filtering by energy? how to filter it well for bijective parameterization?}\justin{will this heuristic mess up the convergence theory?}

\subsection{Convergence on Fixed Lambda}
\label{sec:convergence}

\minchen{[NOTE] superscript in this section is old}

As our method is defined to guarantee convergence to a local optimum, we now analyze the convergence rate. First, $L$ monotonically decreases in each step. Now we look at smooth descent step $i$ and $i+1$, from $L^i \geq L^{i+1}$ we have
\[ E^i_{SD} - E^{i+1}_{SD} \geq \frac{1}{\lambda} (E^{i+1}_{se} - E^i_{se}) \geq \frac{1}{\lambda\sqrt{(\sum_t |A_t|)/\pi}} |e|_{min} \]
if we now only consider splitting operations that keep increasing $E_{se}$. $E_{SD}$ is lower-bounded theoretically by $4$, so we have
\[ n_{alter} \leq \frac{\lambda\sqrt{(\sum_t |A_t|)/\pi}}{|e|_{min}} (E^0_{SD} - 4) \]
The most important hint we can read from this is, to accelerate convergence, we can move through multiple vertices on $G_T$ in each topology descent step to increase $E^{i+1}_{se} - E^i_{se}$.

\minchen{[NOTE] seems not intuitive to generalize to involve merge operations}
\justin{nothing about this discussion seems to involve convergence rate.} \minchen{maybe change "rate" to "speed"?}