% !TeX root = DCSearch.tex

\subsection{Topology Steps}
\label{sec:topologyStep}

In topology steps, we move to a neighboring UV topology with the best first-order reduction in $E_w$ via conducting a basic topological operation searched from the given operation set $\hat{\mathcal{E}}^i_T$. However, if the first-order reduction is not satisfying the threshold $\delta^i$, nothing will be changed.

\begin{algorithm}[h]
\SetAlgoLined
\KwData{$M$, $T^i$, $U^i$, $\hat{\mathcal{E}}^i_T$, $\delta^i$}
\KwResult{$T^{i+1}$, $U_a^{i}$}
\For{each $e^{i,j}_{T}$ in $\hat{\mathcal{E}}^i_T$}{
  compute $\hat{f}_e(e^{i,j}_{T}) = \Big(E^j_s + \lambda \min_{U^{i,j}} E_d(T^j,U)\Big) - E^i_{w}$\;
}
$U_a^{i} \leftarrow U^i$\;
\If{$\min_{e^{i,j}_{T} \in \hat{\mathcal{E}}^i_T} \hat{f}_e(e^{i,j}_{T}) \leq \delta^i$}
{
  $T^{i+1} \leftarrow T^j$, $U_a^{i,j} \leftarrow \argmin_{U^{i,j}} E_d(T^j,U)$\;
}
\caption{Topology Step $j$}
\end{algorithm}

We compute the first-order reduction of $E_w$ from $T^i$ to $T^j$ for all $e^{i,j}_T \in \mathcal{E}^i_T$ in parallel, and only moves to $v^j_T$ with smallest $\hat{f}_e(e^{i,j}_{T})$, if this energy decrease is smaller than $\delta^i$. Recall that the threshold $\delta^i$ is set as the energy decrease in the current descent step. For deciding a search direction, $\delta^i$ is near $0$ as the current descent step converged, while for computing the topology step size, $\delta^i$ is not $0$. This together let our alternating process always pick the search step (continous or discrete) that decrease $E_w$ more to perform.

\paragraph{Operation Filtering}
As the number of vertex that could be splitted is in the scale of $n_p$, and it is obvious that initiating a cut in near-isometric regions won't help improve the UV map much, we filter the vertices to be considered in a split operation by computing the standard deviation of the local individual energy gradients on a vertex, and only consider the top $n_p^{0.6}$ vertices with large deviation. \minchen{[TODO] why not filtering by energy? how to filter it well for bijective parameterization?} For each vertex to be splitted in the candidate set, we evaluate every splitting possibility.

\paragraph{First Order Reduction}
We take the one-ring stencil of the splitted or merged vertex, and optimize $E_d$ on it with the one-ring neighbors fixed. As our energies are normalized, we renormalize the energies defined on the local stencil to be the same measure as the energy defined on the whole UV map.

\paragraph{Corner Merge} We merge the 2 incident boundary edges with averaged position. If element inversion is detected, we apply relaxation method~\cite{Agmon1954Relaxation} for projecting average position to inversion free position to start from, however cases are still there where only move the merged vertex is not enough, for which we will just abandon it.

For boundary vertex that connects to another boundary, we free both the 2 boundary vertices while evaluating the local energy decrease. For merge