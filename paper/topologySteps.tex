% !TeX root = DCSearch.tex

\subsection{Topology Steps for Discrete Search}
\label{sec:topologyStep}

In topology steps, we estimate $f_v(v_{T,j})$ for a filtered set of neighbors of $v_{T,i}$ on a local stencil of $U$ as $\hat{f}_v$ and move onto the neighbor $v_{T,i+1}$ with smallest $\hat{f}_v$.

After the fracture has been initiated, we first go to descent step to run an inner iteration and record the energy decrease $\Delta E_w^j$ and energy $E_w^{j,0}$. Then we evaluate $\hat{f}_v$'s for splitting the tail vertex of the newly initiated fracture along its incident edges. If the one with largest $\hat{f}_v$ satisfies $\hat{f}_v - E_w^{j,0} \leq \Delta E_w^j$, we propagate the fracture along this edge, and run another inner iteration to do another propagation query. If there's no propagation that could benefit more than running the inner iteration, we just run another inner iteration and query the propagation again afterwards.

\begin{algorithm}[h]
\SetAlgoLined
\KwData{Input surface, $U^a_j$, $T_j$}
\KwResult{$U_{j+1}$, $T_{j+1}$}
\For{each neighbor in the candidate set}{
  compute $\Delta E_{SD, l}$ locally\;
  compute $\Delta E_{w,l} = (1 - \lambda_t) \Delta E_{SD,l} + \lambda_t \Delta E_{se}$\;
}
\eIf{$\max (|\Delta E_{w,l}|) \geq |(1-\lambda_t)\Delta E_{SD}^j|$}
{
  $U_{j+1} \leftarrow $, $T_j \leftarrow $\;
}
{
  $U_{j+1} \leftarrow U^a_j$\;
}
\caption{Topology Step $j$}
\end{algorithm}

Even without forward topological line search, an operation is almost always being extended. So we invent the forward line search to speed up the process by avoiding unnecessary descent steps in between the topological operations.

\subsubsection{Candidate Filtering}

\begin{algorithm}[h]
\SetAlgoLined
\KwData{Input model, UV coordinates $U$, UV topology $v_T$}
\KwResult{A filtered set of UV vertices}
\eIf{boundary split}
{
  compute divergence of local gradients for all $n_{v,b}^i$ boundary vertices\;
  pick $(n_{v,b}^i)^{0.8}$ vertices with largest divergence as candidates\;
}
{
  compute divergence of local gradients for all $n_{v,i}^i$ interior vertices that doesn't connect to boundary\;
  pick $(n_{v,i}^i)^{0.8}$ vertices with largest divergence as candidates\;
}
\caption{Candidate Filtering}
\end{algorithm}

\subsubsection{First Order Estimation}

\begin{algorithm}[h]
\SetAlgoLined
\KwData{Input model, UV coordinates $U$, UV topology $v_T$, candidate UV vertices}
\KwResult{new UV topology $v_T$ and UV coordinates $U$}
\For{each candidate UV vertex}{
  \eIf{on boundary}
  {
    \For{each interior incident edge}{
      split and compute $\Delta E_{SD,l}$ locally\;
      compute $\Delta E_{w,l} = (1 - \lambda_t) \Delta E_{SD,l} + \lambda_t \Delta E_{se}$\;
    }
  }
  {
    \For{each pair of incident edges}{
      split and compute $\Delta E_{SD,l}$ locally\;
      compute $\Delta E_{w,l} = 0.5((1 - \lambda_t) \Delta E_{SD,l} + \lambda_t \Delta E_{se})$\;
    }
  }
}
\If{!interiorSplit}
{
  \For{each fracture tail}
  {
    merge the 2 incident boundary edges with averaged position\;
    \If{element inversion is detected}
    {
      project the averaged position to feasible region\;
      \If{feasible region is empty}
      {
        continue\;
      }
    }
    compute $\Delta E_{SD,l}$ locally\;
    compute $\Delta E_{w,l} = (1-\lambda_t)\Delta E_{SD,l}+\lambda_t \Delta E_{se}$
  }
}
conduct the operation with largest $|\Delta E_{w,l}|$
\caption{Local Evaluation}
\end{algorithm}
For boundary vertex that connects to another boundary, we free both the 2 boundary vertices while evaluating the local energy decrease.

\paragraph{Corner Merge} Applied relaxation method~\cite{Agmon1954Relaxation} for projecting average position to inversion free position to start from, however cases are still there where only move the merged vertex is not enough