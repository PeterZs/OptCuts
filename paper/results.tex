% !TeX root = OptCuts.tex

\section{Results and Discussion}
\label{sec:results}

\subsection{Experiments}
\label{sec:results_exp}

\paragraph{Initial Embedding} For disk-topology surfaces, we map the longest boundary to a circle preserving edge length and obtain a Tutte embedding with uniform weights \minchen{[TODO] see whether using MVC weights converge faster}. For closed surfaces and high-genus surfaces, we first apply some simple heuristics to obtain an initial seam, and then treat them as disk-topology surfaces.  
The heuristics for genus-0 surfaces include \justin{``include'' isn't enough; you need to say exactly what you did} farthest point cut and random/curvest one point cut \minchen{[TODO] decide one and add description, or use all?}; for high-genus surfaces, we follow Crane et al.~\shortcite{Crane:2013:DGP} to detect homology generators and then cut along all of them \minchen{[TODO]}.

To show that we search for locally optimal UV maps regardless of the given initial embedding, we run our method starting from triangle soup and preliminary UV maps produced by other methods or by the users. The output maps are still with high quality (Figure~\ref{fig:bad_init_still_ends_well}) \minchen{[TODO]}.

\paragraph{Evaluation Metric} Instead of directly comparing the joint discrete-continuous energy $E_w$ where the balancing factor $\lambda$ is in fact unknown in many methods and there is no reason to set it at any specific values, we compare $E_s$ among results generated by different methods with the same upper bound on $E_d$. \justin{previous sentence should be broken into 2} This metric not only makes it fair for all methods, but also makes sense in practical scenarios, where users mostly want a map with low distortion and as-short-as-possible seams, and they are more sensitive on distortion than seam length. However, since not all methods intuitively support seam placement targeting bounded distortion, we use them to generate several maps for each model and set their output distortion as the upper bound for our method.

\paragraph{Quality and Timing Comparisons} \minchen{[TODO], also provide detailed settings on the compared methods, and how much user assistance was needed for other methods} Since different $\lambda$ provides different energies which are also not comparable with other methods, we directly compare our self-weighted version with other methods. We demonstrate our framework's capabilities by first comparing to AutoCuts~\cite{Poranne2017Autocuts} and two typical classic seam cutting methods~\cite{Gu2002Geometry,Sheffer2002Seamster}. Given the same input surface and initial UV map, we efficiently reach identical distortion bounds with shorter seam lengths (Figure~\ref{fig:QT_comp}). 
When we change the settings to obtain nearly isometric UV maps, the quality of the seams by other methods drop drastically while our method keeps generating high-quality seams (Figure~\ref{fig:strict_bounds_comp}).

They key difference between our framework and two-pass methods is that, our seams are computed to directly improve the given distortion measurement rather than some approximated heuristics. Besides, since the problem is highly nonlinear, seams that benefits one region might be redundant due to the existance of another seam in a near or far region. Our framework is capable of removing those redundancy with our topology search, but the two-pass methods can not.
\minchen{[TODO] add specific analysis to the comparison between ours and the 3 methods}

\paragraph{Triangulation Invariance} \minchen{[TODO] not sure whether applicable}

\paragraph{Scalability} \minchen{[TODO]}

\subsection{Variations}
\label{sec:results_variations}

Without changing the framework, simply reformulating $E_w = E_s + \lambda E_d$ according to different needs enables OptCuts to solve mesh parameterization problems in many variations:

\paragraph{Global Bijectivity} \danny{suggest we no longer consider this a variation and instead use it as a key part of our full algorithm - see my comments earlier.} \minchen{[DOING]} Augmenting our $E_d$ with a collision handling energy $E_b$ will easily achieve joint seam placement and bijective mesh parameterization. We show that by adding a scaffold mesh~\cite{Jiang2017Simplicial} to the voided regions of the UV map and preventing the scaffold mesh from degenerate, our method automatically generate high-quality bijective maps with optimal seams different from that of locally injective parameterization (Figure~\ref{fig:bijective_vs_injective}). When computing $\hat{f}_e$, besides also including the one-ring triangles on the scaffold mesh for computing energy decrease, we also need to move the splitted vertices slightly apart to leave room for inserting new scaffold mesh triangles.

\paragraph{Conformal Parameterization} \minchen{[TODO]} Using a conformal energy~\cite{Hormann2000MIPS,Sheffer2005ABFPP} for $E_d$ will achieve joint seam placement and conformal parameterization. Figure~\ref{fig:conformal_vs_isometry} shows some results with $E_d = E_{ABForMIPS}$~\cite{} compared to results with $E_d = E_{SD}$, where different seams are generated while our framework stays the same.

\paragraph{Regional Seam Placement} \minchen{[TODO]} On the discrete side, if we reweight $E_{SL}$ with an edge prior provided by the user or an algorithm~\cite{} as
\[ E_s = \hat{E}_{SL} = \sum_{i\in\mathcal{S}} w_{SL,i} E_{SL,i} \quad w_{SL,i} \in \mathcal{R^+} \]
we could bias the seam placement towards regions e.g. where continuity is less in demand (Figure~\ref{fig:regional_seam_placement}).

\minchen{[TODO] add specific analysis to the results}