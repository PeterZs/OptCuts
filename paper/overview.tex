% !TeX root = OptCuts.tex

\section{Framework Overview}

\paragraph{A Discrete-Continuous Problem}

Given an input triangle mesh of a 3D surface with $n_v$ vertices and its initial UV map with topology $T_0$ and coordinates $U_0 \in \mathcal{R}^{2n_v}$, we automatically search for
\[ \argmin_{T,U} E_w(T,U) \]
where $E_w(T,U) = E_s(T,U) + \lambda E_d(T,U)$ is the total energy combining both seam energy $E_s$ and distortion energy $E_d$ with a balancing factor $\lambda \in \mathcal{R^+}$. Notice that usually $E_d$ is smooth, while $E_s$ is defined to be non-differentiable, which gives us a discrete-continuous problem.

\paragraph{Joint Discrete-Continuous Search}

Instead of trying to solve an ill-conditioned real-valued optimization by approximating $T$ with $U$ \cite{Poranne2017Autocuts}, we alternatingly optimize $T$ (in topology steps) and $U$ (in descent steps), augmenting the continuous search with a discrete topology search.

Descent steps deal with the continuous search
$\min_U E_d$
where $T$ is fixed and Newton-type method is applied (Section~\ref{sec:descentStep}).
Topology steps deal with the discrete search (Section~\ref{sec:topologyStep}) where a neighboring topology is searched based on the first-order reduction it would cause in $E_w$.
While each descent step is one complete Newton-type iteration with back-tracking line search, a sequence of topology steps alternated with descent steps are used to perform a forward line search in topology space. \minchen{analogously a direction is decided first and followed with elongations}

Given a topology change, we evaluate the first-order reduction of $E_w$ it would cause by minimizing $E_d$ only on the local stencil of the edited seam edge and combine it with the $E_s$ change. Experiments demonstrate that this is efficient and sufficient for always editing the seams reasonably and sustainably (Section~\ref{sec:results_exp}).

Since we ensure $E_w$ to be monotonically decreased in both the descent steps and topology steps, we can easily prove that a near-stationary point can be reached for any input within a bounded number of alternations (Section~\ref{sec:convergence}).

\paragraph{Self-Weighted Objective}

In order to automatically decide $\lambda$ and enable users to directly control the distortion they expect for the output UV map, we formulate a constrained optimization seeking to minimize seam energy $E_s$ subject to user-specified distortion bounds $b_d$:
\[ \min_{T,U} E_s(T,U) \quad s.t. \quad E_d(T,U) - b_d \leq 0 \]
Introducing $\lambda$ as a dual variable gives us
\[ \min_{T,U} \max_{\lambda \in \mathcal{R^+}} E_s(T,U) + \lambda(E_d(T,U) - b_d) \]
We solve it by adding a regularization term to make it smooth for $\lambda$, and alternatingly updating dual variable $\lambda$ and primal variables $T$ and $U$ (Section~\ref{sec:self_weighting}).

The challenge here is that the initial embedding we start from is usually with high distortion, thus infeasible, and the quality of the output seams is highly affected by the $\lambda$ updating scheme. \minchen{[TODO] briefly describe what we do after finalized}
 
\paragraph{Variations}

When defining the total energy $E_w(T,U) = E_s(T,U) + \lambda E_d(T,U)$ for our discrete-continuous problem, the continuous energy $E_d(T,U)$ can be any kinds of distortion energy, measuring either isometry or conformality, and it can also be augmented with energies enforcing extra features such as bijectivity or seamlessness. This is also true for the discrete energy $E_s(T,U)$, which can be any energy related to seams, such as length, regional preference, smoothness, etc. The key is that with all these variations (we show some of them in Section~\ref{sec:results_variations}), our framework stays the same, and it efficiently generates optimal seams for the specific $E_w$. Moreover, we believe our framework can be applied to other discrete-continuous geometry processing problems, where $E_s$ can be any energy related to mesh topology, and $E_d$ can be any energy defined on the surface geometry.
