\appendix

\section{Auto-mode AutoCuts}
\label{app:autoauto}

Recall that in the AutoCuts model the topology of the UV mesh does not change: it remains a triangle soup. Instead, the seam-penalty energy that pulls triangles together is parameterized by a $\delta$ parameter which indicates how far vertices must be from one another to be considered truly disconnected by a seam. Finding a single weighting $\lambda$ that works well for all geometries poses a challenge and likewise $\delta$ requires adjustment over iterations depending on the current stage of the optimization. Thus AutoCuts is generally most effective in interactive mode with a user in-the-loop. For instance, starting from a disconnected triangle soup, a user will gradually decrease $\delta$ defining a {\em homotopy path} over the course of optimization to arrive at a final set of seams. The user also needs to move parameterized components to guide the UV layout towards a globally bijective solution. 

Here the key step in formulating the automated mode for AutoCuts is in automating the homotopy path. This requires prescribing an automated sequence of updates to $\delta$ with a uniform set of mesh-adaptive parameters. In final we set the Autocut parameter $\lambda = 0.4$ and start with $\delta_0=100\xi^2$. We then half this value at the start of each homotopy iteration $\delta_i = \frac{1}{2}\delta_{i-1}$ until it reaches $\delta_\text{term}=10^{-4}\xi^2$. Inside each homotopy iteration, we detect convergence by setting a tolerance $\sqrt{3n_t}\times10^{-3}\xi$ on the $L^2$ norm of UV coordinate changes. Here $\xi$ is the characteristic value of the average edge length in the mesh while $n_t$ is the number of triangles---we seek to make these automated measures robust over varying mesh-resolutions as well as scale.