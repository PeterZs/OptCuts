% !TeX root = OptCuts.tex

\section{Introduction}
\danny{I reorganized this section.}
% context
Mesh parameterization is a critical task in computer graphics, with applications including texture mapping, remeshing, and detail transfer.  At its core, parameterization seeks to find a mapping from a surface---typically represented as a triangle mesh---into the plane.  To do so, parameterization tools must cope with intertwined challenges in topology and geometry:  A high-quality parameterization cuts a surface into simple disk-shape patches so that each can be mapped into the plane with a reasonably small level of stretch or compression.

Given its broad applicability, it comes as no surprise that parameterization has been a focus of research in geometry processing.  Algorithms in this domain focus on two key aspects of the problem.  Particularly well-studied are \emph{geometric} techniques that assume a surface has already been cut into disk topology and now needs to be mapped into the plane while maintaining fixed connectivity; at this point parameterization becomes a real-valued optimization problem to preserve angles and areas while maintaining local or global injectivity. In parallel with these, \emph{topological} algorithms find reasonable seams partitioning a surface into individual pieces that can be parameterized.  

With a few notable exceptions, previous work treats seam placement and fixed-topology mapping as separate steps in the parameterization pipeline.  This is a key drawback, as seam placement and complexity have strong bearing on the best \emph{possible} geometric parameterization of each patch.  Coupling these problems requires navigating between two extremes:  A zero-distortion but topologically complex parameterization maps every triangle into the plane independently, while a high-distortion but topologically simple mapping might only require puncturing a few points in the input surface.

% task and object
In this paper, we propose a parameterization method that couples topology and geometry named 
{\em OptCuts}.
OptCuts is 
a joint discrete-continuous optimization framework that progressively updates seam placement between distortion minimization steps to search for the best UV map. 
%
We measure the trade-off between topological and geometric considerations by balancing distortion measure of choice, e.g. the symmetric Dirichlet energy~\cite{Smith2015Bijective}, with normalized seam length, reaching a parameterization that is near-stationary with respect to both seam placement and coordinate positions within a bounded number of iterations.
%\minchen{[NOTE] (Here stationary w.r.t. UV topology is only in the approximation sense, because there might still be basic topological operations that could decrease the objective but end up not chosen because it is filtered out or its locally evaluated energy decrease is not the largest one.)}

% self-weighting
%Although we acquire adaptivity to various inputs by normalizing the energy, %<-- not sure what this means
Measuring a balance between distortion and seam length, however, has previously required a choice of a relative scaling factor between these two objectives.  From a practical perspective, it is difficult for users to choose such a coefficient. %trading off between the two. 
Likewise, in practice, for high-quality solutions this balancing factor must be somehow be updated, often by user intervention, within the parametrization process\ \cite{Poranne2017Autocuts}. Instead we construct our joint optimization as a \emph{constrained} parametrization problem. We propose to find locally minimal seam length for any user-specified distortion bound. Setting distortion bounds as a hard constraint preserves quality, while simultaneously enabling the exploration of optimal seams satisfying the bound. In turn our scaling factor directly falls out from our model as the Lagrange multiplier of the distortion bound and is thus adpatively updated directly by our optimization method.
%(Section~\ref{sec:self_weighting}).

% challenge
In contrast to previous work that builds up a parameterization from disconnected triangle soup~\cite{Poranne2017Autocuts} or by cutting between points of maximal predicted distortion~\cite{Gu2002Geometry,Sheffer2002Seamster}, our constrained optimization  method minimizes with seams combinatorially, directly extending real-valued distortion minimization to the regime where seams can be introduced and removed. As distortion s typically reduced by line search in UV coordinate space, we accompany this with a secondary search procedure in ``topology space:'' We alternate between reducing stretch and introducing potential topological changes based on the first-order reduction they would cause in the summed distortion and seam length terms of the objective. %in each topology search iteration we first find a search direction that locally decrease the objective the most (in topology step, Section~\ref{sec:topologyStep}); and then we conduct a newly derived forward-tracking line search scheme alternated with distortion minimization iterations to decide the step-size (in descent step, Section~\ref{sec:descentStep}).% too much detail for intro


% experiments
We demonstrate our method's capabilities by comparing to AutoCuts~\cite{Poranne2017Autocuts} as well as typical classic seam cutting methods~\cite{Gu2002Geometry,Sheffer2002Seamster}. % (Section~\ref{sec:results}). 
Given the same initial UV map, we efficiently reach identical distortion bounds with shorter seam lengths. We test on large-scale inputs to demonstrate scalability, and use inputs with same shape but different triangulation to show that our method is stable to changes in mesh structure.

% contribution
\paragraph*{Contributions.} 
We present a novel framework that jointly optimizes seam placement and distortion for mesh parameterization, \minchen{strengthen that "seam placement" refer not only to length, but also location? and distortion can be various kinds of distortion?}. To do so we construct a constrained formulation that enables users to optimize UV maps for specified distortion bounds while gaining locally minimal seam length without manual intervention (for e.g. changing scaling terms or regularizers). Key to our method are topological search strategies as well as a dual variable treatment searching for optimal cuts with bounded distortion. Our framework is easily extended to handle global bijectivity \danny{We should update what we want to say here about bijectivity - seems like there's a nice observation regrading how bijectivity actually makes the optimization both better and easier but we should be clear if this is the case - if so bijectivity is more than an "ad-on" and should be a first class citizen.} and seamlessness. \justin{what do you mean by seamlessness?} \minchen{I was thinking about global or seamless parameterization. What would be the most practical "seamless" property a user may want?} \vova{since we don't plan to implement seamless parameterization, we can discuss it in future work. } We also demonstrate that our approach applies to other discrete-continuous geometry processing problems \danny{Don't think we will have time nor need to do this last item - instead again something good for future work discussion as well}.% (Section~\ref{sec:conclusion}).
