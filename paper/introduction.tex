% !TeX root = OptCuts.tex

\section{Introduction}
% context
Mapping three-dimensional meshes to the plane is a fundamental task in computer graphics.  The two-dimensional mesh embeddings produced by mapping methods are commonly used to store reflectance functions, normals, and displacements
for the mesh, providing a domain for painting, synthesizing, and manipulating texture and geometric details. 
%Mesh parameterization is a critical task in computer graphics, with applications including texture mapping, remeshing, and detail transfer.  
%\vova{I removed non-texture apps for now, because I'm not sure if our formulation is as easy to motivate for remeshing and detail transfer.}
%
The usability of these embeddings is highly dependent on two interconnected factors: the surface distortion introduced by the mapping and the length of the surface cuts, forming seams across which the mapping is discontinuous~\cite{Sheffer07_ParameterizationSurvey,Hormann2008}. Both high distortion and longer seams are detrimental to downstream applications. Yet, reducing distortion below a desired 
%some %intrinsic surface %<--- what does this mean
bound typically requires introducing longer seams. 
% To do so, parameterization tools must cope with intertwined challenges in topology and geometry:  a high-quality parameterization must cut a surface into simple %disk-shape 
%patches so that each can be mapped to the plane with a reasonably small level of distortion.

Given its broad applicability, parameterization has long been a focus of research in geometry processing. Algorithms in this domain focus on these two key aspects of the problem \cite{Sheffer07_ParameterizationSurvey,Hormann2008}.  Particularly well-studied are \emph{geometric} techniques that assume a surface has already been cut into disk-topology segments %charts %<--- charts are the maps
 that each then need to be mapped into the plane with minimal distortion while maintaining fixed connectivity; at this point, parameterization becomes a real-valued optimization problem that seeks to minimize changes in mesh angles and areas while maintaining local or global injectivity. Complementing these techniques, \emph{topological} algorithms find reasonable seams, either keeping the surface in one piece or partitioning it into individual segments that can then be parameterized with low distortion (Section~\ref{sec:related}).   

In contrast, we propose a joint optimization algorithm {\em OptCuts} that simultaneously optimizes for both seam length and the corresponding distortion of the embedding.
Our algorithm is based on our proposed minimization model problem that directly and automatically balances our optimization between seam length and parametric distortion measures. Manually balancing distortion and seam quality requires a choice of a relative scaling factor between these two objectives. From a practical perspective, it is difficult for users to choose this factor as the two terms measure very different quantities and no such setting can provide a guarantee on the quality of the generated map's distortion. 
On the other hand, users typically have a clear sense of the amount of distortion they consider acceptable for their application. Motivated by this observation we cast our coupled seam and distortion optimization as a \emph{constrained} problem to find charts with locally minimal seam lengths that strictly satisfy a user-set distortion bound. Treating the distortion bound as a hard inequality constraint guarantees a pre-specified level of mapping quality, while enabling the exploration of optimal seams satisfying this bound. %In turn, o %<--- not sure what ``in turn'' means in this context

Prior methods coupling distortion reduction and seam computation have generally required hand-tuning a number of user-exposed parameters and, as in the recently proposed AutoCuts~\cite{Poranne2017Autocuts}, even advocate manual intervention by interactively adjusting these parameters during an optimization's iterations.

In contrast, OptCuts is a fully automatic optimization method: users provide their desired distortion bound and OptCuts then directly computes a parametrization satisfying this bound with locally minimal edge lengths. Maps provided are always locally injective and, as we will show, can additionally be constrained to be bijective and even support additional, user-provided seam constraints and biases when desired. Likewise, as demonstrated by our comparisons in Section~\ref{sec:results}, Figures~\ref{fig:autocut} and ???~\danny{update when all comparison figures and tables are in.}, when compared to previous methods that do provide an automatic mode~\cite{BoundedDistortParam:2002,Poranne2017Autocuts} OptCuts produces much shorter seams when its bound is set for the same achieved distortion. Even in comparing to hand-tuned UV maps, we find that OptCuts consistently finds similar seam lengths for comparable or even better distortion bounds; see Section~\ref{sec:results}, Figures ??? \danny{update when all comparison figures and tables are in.}. Finally, as we show in Section~\ref{sec:results}, Figures ??? \danny{update when all comparison figures and tables are in.}, OptCuts can also be used to polish any preexisting UV map irrespective of the method used to create it. OptCuts can take an arbitrary UV map as input and improve either seam length while preserving the current distortion bound or even improve upon distortion as well, by setting a lower distortion bound.

  To achieve these gains we begin by casting global parameterization as the constrained minimization formulated with seam length as our objective and a distortion bound as our inequality constraint. We then observe that the \emph{Lagrangian} of this constrained minimization's saddle-point problem directly retrieves the multi-objective optimization formed by the weighted sum of seam length and map distortion. However, the key observation here is that now there is now a natural scaling between the two terms that is directly defined by the Lagrange multiplier of distortion bound. Regularization of iterated updates to this multiplier then allows us to smoothly explore variations of the Lagrangian over the space of seam cuts. 
  
  Next, we observe that in order to solve this saddle-point problem we must optimize over both smooth vertex parameters and discrete changes in topology. Exhaustive search is clearly not an option. Instead, we propose a discrete-continuous optimization method that explores decrease of distortion and seam length over both classical, smooth descent directions and along propagations of topological merging and cutting operations on the UV mesh. When desired, we additionally enforce additional constraints to achieve globally bijective maps. Finally, we additionally allow UV artists to guide seam placement away from salient regions by enabling painting over the surface. OptCuts then avoids seam placement in the regions in proportion to the intensity of the painting.
  
  Together, these components form the core of our OptCuts algorithm. Over a wide range of examples we show that OptCuts efficiently achieves all attempted distortion bounds while locally minimizing seam length for both locally injective and bijective mappings. In Section~\ref{sec:results} we compare against both state of the art algorithms and industrial UV-parameterization tools and show that for the same achieved distortion bound, we consistently improve seam-length over prior automated methods, while our automated results closely match with the results of hand-tuned methods. We also evaluate OptCuts over a large benchmark of parametrization problems, demonstrating that across mesh scales and problem difficulties OptCuts successfully obtains user-specified distortion bounds while efficiently minimizing seam length. 
  
%% contribution
% contribution
\subsection{Contributions.}

In summary we propose OptCuts, to our knowledge the first fully automated global parameterization algorithm that obtains bijective maps satisfying prescribed distortion bounds while locally minimizing seam length. To do so we first formulate a new, simple-to-state, constrained seam-length minimization model problem. We then solve our problem by our proposed discrete-continuous algorithm for the saddle-point problem using a combined discrete search over propagated mesh operations and smooth descent over vertex positions. We evaluate OptCuts to show efficient performance and scaling. Across a wide range of automated methods it improves over the state of the art, while automatically obtaining comparable quality results to hand-tuned parameterization methods. 
