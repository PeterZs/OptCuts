% !TeX root = OptCuts.tex

\section{Optimization Framework}

To solve the saddle-point problem in Eq.~\ref{eq:p2}, we alternately improve both our primal variables $(T, U)$ and our dual variable $\lambda$ starting from an initial UV map $(T^0, U^0)$ and $\lambda^0 = 0$ (Algorithm~\ref{alg:selfWeight}).

\begin{algorithm}[!h]
\SetAlgoLined
\KwData{$M$, $T^0$, $U^0$, $b_d$}
\KwResult{$T^*$, $U^*$}

$\lambda^0 \leftarrow 0$, $k \leftarrow 1$\;

\Do{either primal or dual not converged}{
  $\lambda^{k}$ $\leftarrow$ dualUpdate($T^{k-1}$, $U^{k-1}$, $b_d$, $\lambda^{k-1}$); // Section~\ref{sec:dualUpdate}\\

  $(T^{k}, U^{k})$ $\leftarrow$ primalUpdate($M$, $T^{k-1}$, $U^{k-1}$, $\lambda^{k}$); // Section~\ref{sec:primalUpdate}\\

  $k \leftarrow k + 1$\;
}
$(T^*, U^*)$ $\leftarrow$ $(T^{k-1}, U^{k-1})$\; 

\caption{OptCuts}
\label{alg:selfWeight}
\end{algorithm}

\subsection{Initialization}
To obtain an initial UV map for an input surface, we map its initial seam to a circle preserving edge length and parameterize the rest of the vertices through Tutte embedding with uniform weights.

We compute initial seams for different surfaces according to their topology and geometry. For disk-topology surfaces, we simply pick their longest boundary as the initial seam. For genus-0 closed surfaces, we randomly pick 2 connected edges as the initial seam. \minchen{[TODO] change to curvest one point cut or farthest point cut if they are better} For high-genus surfaces, we follow Crane et al.~\shortcite{Crane:2013:DGP} to detect homology generators and connect all of them as the initial seam \minchen{[TODO]}.

We simply start by ignoring the distortion constraints with $\lambda$ set to $0$, and let our dual update to modify $\lambda$ according to the intermediate distortions.

\subsection{Dual Update}
\label{sec:self_weighting}
\label{sec:dualUpdate}

% Our overall minimization is inequality constrained with a specified upper bound $b \in \mathbb{R}_+$ on distortion. \justin{I moved a parenthetical to a Minchen comment assuming he'll write it more formally}\minchen{(L2 norm on SD  energy for now - pretty easy to modify to an extremal measure if we want later on.)}

%
In this section we describe the update for the dual variable $\lambda^k$ given the current topology and embedding $T^{k-1}, U^{k-1}$.

The inner objective\minchen{Danny: which?} in the saddle-point problem (Eq.~\ref{eq:p2}) is nonsmooth in $\lambda$ since it does not take into account the fact that we might start away from feasibility and want to iteratively improve both our primal variables $(T, U)$ and our dual variable $\lambda$. To smoothly update to a current $\lambda^{k}$ in iteration $k$ from a previous estimate $\lambda^{k-1}$, we add a simple quadratic regularizer $R(\lambda,\lambda^{k-1}) = \frac{1}{2\kappa} (\lambda- \lambda^{k-1})^2$ to make sure $\lambda$ iterates behave reasonably. (We simply set $\kappa = 1$.)

For iteration $k$ this gives us 
\[ \min_{T,V} \max_{\lambda \geq 0} E_{s}(T^{k-1}) + \lambda \big( E_{d}(T^{k-1}, U^{k-1}) - b_d\big) - \frac{1}{2\kappa} (\lambda- \lambda^{k-1})^2 \]
which can be solved in closed form as
\[ \lambda^{k} \leftarrow \max\big(0,\kappa \big( E_{d}(T^{k-1}, U^{k-1}) -b \big) + \lambda^{k-1}\big) \]

\danny{(Notice that throughout the above we can define a progressive $\lambda$ without needing to employ subgradients to reason about nonsmoothness in our sparsity energy.)}\justin{didn't follow this, not sure it's needed}

\justin{Can you call the lambda update a proximal step?}


\subsection{Primal Update}
\label{sec:primalUpdate}

Our primal update is a search procedure towards minimizing \eqref{eq:L} for a fixed $\lambda$ starting from the current UV map.
Rather than trying to solve a potentially ill-conditioned real-valued optimization by approximating $T$ with $U$\ \cite{Poranne2017Autocuts}, we alternate local optimizations over $U$ and then $T$ (Algorithm~\ref{alg:DCSearch}).
\vova{I think we need to explain what's making it ill-conditioned. Also, I think it's unnecessary to cite autocuts here...}

\begin{algorithm}[h]
\SetAlgoLined
\KwData{$M$, $T^{k}$, $U^{k}$, $\lambda^{k+1}$}
\KwResult{$T^{k+1}$, $U^{k+1}$}
$i \leftarrow 1$, $T^{k,0} \leftarrow T^{k}$, $U^{k,0} \leftarrow U^{k}$\;
$\hat{\mathcal{E}}^{k,0}_T \leftarrow$ operationFiltering(), $\delta^{k,0} \leftarrow 0$\;
\Do{smooth descent step not converged}
{
	($T^{k,i}$, $U_a^{k,i-1}$) $\leftarrow$ topologyDescentStep($M$, $T^{k,i-1}$, $U^{k,i-1}$, $\hat{\mathcal{E}}^{k,i-1}_T$, $\delta^{k,i-1}$, $\lambda^{k+1}$); // Section~\ref{sec:topologyStep}\\
	($U^{k,i}$, $\delta^{k,i}$) $\leftarrow$ smoothDescentStep($M$, $T^{k,i}$, $U_a^{k,i-1}$); // Section~\ref{sec:descentStep}\\
	$\hat{\mathcal{E}}^{k,i}_T \leftarrow$ operations for extension\;
	$i \leftarrow i+1$\;
}
$T^{k+1} \leftarrow T^{k,i-1}$, $U^{k+1} \leftarrow U^{k,i-1}$
\caption{Primal Update $k+1$}
\label{alg:DCSearch}
\end{algorithm}

Our smooth descent steps apply standard continuous search. We locally minimize distortion $E_d$ over vertices $U$ while holding topology, $T$, fixed detailed in Section~\ref{sec:descentStep}.
%
Our topology descent step then applies a discrete search over local topology changes. We search neighboring topology to maximize a first-order reduction in the Lagrangian $L$ detailed in Section~\ref{sec:topologyStep}.

Similar to continuous search, topology search (Section~\ref{sec:topologySearch}) is also composed of two stages: First decide a search direction (initiating a local seam edit), and then compute a proper step size (extending the edit) to ensure sufficient energy decrease. While each smooth descent step applies a complete Newton-type iteration with backtracking line search, we interleave topology descent steps with smooth descent steps to perform a forward line search in topology space - their step size is \emph{increased}, rather than backtracked. This makes each primal update a single topology search step.

There is no point in searching for every neighboring UV topology since, for example, initiating a cut in near-isometric regions will not help improve the map much. Thus we have a filtered candidate operation set $\hat{\mathcal{E}}^{k,i-1}_T$ for each topology descent step $(k+1,i)$. When deciding a direction, we fill the set with all promising operations produced by operation filtering (Section~\ref{sec:operationFiltering}), while when computing the step size, we fill the set with only the same operation as in the previous topology descent step for extending the step size.

In total, by ensuring monotonic decrease in $L$ over both the smooth and topology descent steps, we then prove that a near-stationary point can be reached for any input within a bounded number of alternations (Section~\ref{sec:convergence}). \danny{for varying $\lambda$ - i.e. our constrained problem seems like we need to spend a bit more work on this - see comments below.}

% \vova{The section below seem to be concerned with convergence of Alg 1, which is beyond self-weighted objective, IMHO... 
% It is also confusing that there a subsection 5.7 on convergence...}
\subsection{Convergence}

Before and after each primal update, our continous search is under convergence. If at the first inner iteration of a primal update there is no neighboring topology that could reduce the multi-objective in the first-order manner, we say our discrete topology search also converges, so does the primal update, and the UV map will be the minimizer of the multi-objective $L$.

Theoretically, the dual update converges in two situations: either at $\lambda^* = 0$ when the primal minimizer $(T^*, U^*)$ is inside the feasible region ($E_d(T^*, U^*) < b_d$) with no seams, or at some $\lambda^* > 0$ with $(T^*, U^*)$ on the boundary of the feasible region ($E_d(T^*, U^*) = b_d$) \cite{a computational optimization book}. However, for the letter case, since there is only a finite number of configurations of $T$, the near stationary $E_d$'s may not be exactly equal to $b_d$. Instead of setting a relatively large tolerance for detecting convergence, we stop when it first tries to violate distortion bound $b_d$ from being feasible.
