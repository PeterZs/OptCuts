% !TeX root = OptCuts.tex

\section{Optimization Framework}

To solve our constrained optimization (\ref{eq:p1}) we focus on resolving the saddle-point problem (\ref{eq:p2}). 
We start from an initial, valid UV map $(T^0, U^0)$ and bootstrap our dual variable with $\lambda^0 = 0$. OptCuts then iteratively alternates between primal solves to improve geometry, $(T, U)$, and dual solves to update our multiplier, $\lambda$, to encode the new balancing term between distortion and seam quality for the next primal solve we take. See Algorithm~\ref{alg:selfWeight}.

\begin{algorithm}[!h]
\SetAlgoLined
\KwData{$M$, $T^0$, $U^0$, $b_d$}
\KwResult{$T^*$, $U^*$}

$\lambda^0 \leftarrow 0$, $k \leftarrow 1$\;

\Do{either primal or dual not converged}{ 
  $\lambda^{k}$ $\leftarrow$ dualUpdate($T^{k-1}$, $U^{k-1}$, $b_d$, $\lambda^{k-1}$); // Section~\ref{sec:dualUpdate}\\

  $(T^{k}, U^{k})$ $\leftarrow$ primalUpdate($M$, $T^{k-1}$, $U^{k-1}$, $\lambda^{k}$); // Section~\ref{sec:primalUpdate}\\

  $k \leftarrow k + 1$\;
} 
$(T^*, U^*)$ $ \leftarrow$ $(T^{k-1}, U^{k-1})$\; 

\caption{OptCuts}
\label{alg:selfWeight}
\end{algorithm}

%\danny{Suggest moving this to our implementation section 
%\subsection{Initialization}
%To obtain an initial UV map for an input surface, we map its initial seam to a circle preserving edge length and parameterize the rest of the vertices through Tutte embedding with uniform weights.
%
%We compute initial seams for different surfaces according to their topology and geometry. For disk-topology surfaces, we simply pick their longest boundary as the initial seam. For genus-0 closed surfaces, we randomly pick 2 connected edges as the initial seam. \minchen{[TODO] change to curvest one point cut or farthest point cut if they are better} For high-genus surfaces, we follow Crane et al.~\shortcite{Crane:2013:DGP} to detect homology generators and connect all of them as the initial seam \minchen{[TODO]}.
%
%We simply start by ignoring the distortion constraints with $\lambda$ set to $0$, and let our dual update to modify $\lambda$ according to the intermediate distortions.


\subsection{Primal Update}
\label{sec:primalUpdate}
Our $k$th primal update is a joint discrete-continuous search over variations in geometry to minimize the weighted sum of seam length and distortion energies in our Lagrangian (\ref{eq:L}). We hold the current iterate's dual variable, $\lambda^k$, fixed and initialize with the last iterate's geometry $(T^{k-1}, U^{k-1})$.

%Rather than trying to approximate $T$ with non-smooth energies of $U$ and solve a potentially ill-conditioned real-valued optimization, we alternate local optimizations over $U$ and then $T$ in each inner iteration and obtain $(T^{k+1}, U^{k+1})$ when both are converged (Algorithm~\ref{alg:DCSearch}).

We initially experimented with solving our primal problem by approximating topology change with non-smooth energies on duplicated vertices of the input mesh; i.e., in a manner similar to AutoCuts\ \cite{Poranne2017Autocuts}. Unfortunately, this quickly led to problems with strongly ill-conditioned real-valued optimization and prevented scaling to larger meshes. 

To counter these challenges we directly optimize in alternating inner steps searching over changes in topology $T$ and then vertices $U$. These inner iterations loop until we 
%converge to a minimizer of the Lagrangian at the fixed $\lambda^k$, 
we reach a stationary point, 
giving us iterate $k$'s newly updated geometry $(T^{k}, U^{k})$ (see Algorithm~\ref{alg:DCSearch}).% and Section\ \ref{sec:topologySearch}. 

%\danny{Minchen: correct me if I'm wrong but I don't believe you employ an actual convergence measure - you simply wait until the $(T,U)$ remain unchnged by further iterates. Is this correct?} \minchen{Correct. We simply stop the discrete-continuous alternations when continuous search converges since it also means topology search won't go further in the current direction. For continuous search, besides gradient tolerance, I also have a relative energy decrease tolerance ($10^{-6}$). It is not because we have troubles converging to small gradient, which is actually not necessary. Setting extra tolerance on relative energy decrease is more stable than using larger gradient tolerance.}

\begin{algorithm}[h]
\SetAlgoLined
\KwData{$M$, $T^{k}$, $U^{k}$, $\lambda^{k+1}$}
\KwResult{$T^{k+1}$, $U^{k+1}$}
$i \leftarrow 1$, $T^{k,0} \leftarrow T^{k}$, $U^{k,0} \leftarrow U^{k}$\;
$\delta^{k,0} \leftarrow 0$\;
\Do{either step is not converged}
{
	($T^{k,i}$, $U_a^{k,i-1}$) $\leftarrow$ topologyDescentStep($M$, $T^{k,i-1}$, $U^{k,i-1}$, $\delta^{k,i-1}$, $\lambda^{k+1}$); // Section~\ref{sec:topologySearch}\\
	($U^{k,i}$, $\delta^{k,i}$) $\leftarrow$ smoothDescentStep($M$, $T^{k,i}$, $U_a^{k,i-1}$); // Section~\ref{sec:descentStep}\\
	$i \leftarrow i+1$\;
}
$T^{k+1} \leftarrow T^{k,i-1}$, $U^{k+1} \leftarrow U^{k,i-1}$
\caption{Primal Update $k+1$}
\label{alg:DCSearch}
\end{algorithm}

Each vertex step of the primal update performs a single iteration of Newton-type, smooth descent with line-search towards minimizing our distortion energies over vertices $U$ while holding topology, $T$, fixed. As our distortion energies, including symmetric Dirichlet, are generally nonconvex we employ the projected-Newton\ \cite{Teran2005Robust} approximation of the Hessian. This is then coupled with 
%For more details see Section \ref{sec:imp}.
search of topology changes to form a customized discrete-continuous topology search method. We defer discussion of the details of this component to Section~\ref{sec:topologySearch}.
%

%Each vertex update step performs a single iteration of Newton-type, smooth descent with line-search towards minimizing our distortion energies over vertices $U$ while holding topology, $T$, fixed. As our distortion energies, including symmetric Dirichlet, are generally nonconvex we employ the projected-Newton\ \cite{Teran2005Robust} approximation of the Hessian here. For more details see Section \ref{sec:imp}.
%%
%%The continuous search is peformed in each smooth descent step, where we conduct a complete Newton-type iteration with backtracking line search towards minimizing distortion $E_d$ over vertices $U$ while holding topology, $T$, fixed (Section~\ref{sec:descentStep}).
%%
%Our topology step requires the construction of a custom discrete topology search method. We defer discussion of the details of this component to Section~\ref{sec:topologySearch}.
%%
%The discrete topology search is performed in each topology descent step by querying a set of neighboring topologies and changing to the one with largest first-order reduction in $L$ if this reduction is prominent (Section~\ref{sec:topologySearch}). 
%

%By thresholding topology changes using the energy decrease of the last smooth descent step, we always proceed either the discrete topology search or the continuous search that decreases $L$ more.\justin{didn't follow previous sentence.}

\subsection{Dual Update}
\label{sec:self_weighting}
\label{sec:dualUpdate}

% Our overall minimization is inequality constrained with a specified upper bound $b \in \mathbb{R}_+$ on distortion. \justin{I moved a parenthetical to a Minchen comment assuming he'll write it more formally}\minchen{(L2 norm on SD  energy for now - pretty easy to modify to an extremal measure if we want later on.)}

%

%In this section we describe the update for the dual variable $\lambda^k$ given the current UV map $(T^{k-1}, U^{k-1})$.


Our Lagrangian in (\ref{eq:p2}) is nonsmooth in $\lambda$. When we exceed the distortion bound, i.e., $E_d(T,U) > b_d$, we have $\lambda = \infty$; when on boundary of the feasible set, $E_d(T,U) = b_d$, we have a finite $\lambda \in \mathbb{R_+}$; and finally, in the strict interior of the set of feasible distortions, we have $\lambda = 0$. While these conditions nicely characterize optimality, we need a way to iterate on $\lambda$ towards the solution in a smooth and robust manner irrespective of whether we are locally exploring a feasible or infeasible distortion. 

At iteration $k$ we thus smoothly approximate the multiplier's behavior by adding a simple quadratic regularizer to the Lagrangian. This keeps the updated multiplier \emph{proximal} to the previous iterate's estimate via Powell's extension\ \cite{}~\danny{Add citation: M.J.D. Powell, On Search Directions for Minimization Algorithms, Mathematical Programming, 4 1973.} of the Augmented Lagrangian to the inequality constrained setting,
%\begin{align}
%	\min_{T,V} \max_{\lambda \geq 0} E_{s}(T^{k-1}) + \lambda \big( E_{d}(T^{k-1}, U^{k-1}) - b_d\big) - \frac{1}{2\kappa} (\lambda- \lambda^{k-1})^2. 
%\end{align} 
\begin{align}
\label{eq:al}
	\min_{T,V} \max_{\lambda \geq 0} E_{s}(T) + \lambda \big( E_{d}(T, U) - b_d\big) - \tfrac{1}{2} (\lambda- \lambda^{k-1})^2. 
\end{align} 
At each iteration, to find $\lambda^k$ we simply fix vertices and topology to  $(T^{k-1},U^{k-1})$. Optimality of (\ref{eq:al}) then gives us our corresponding dual update in closed form
%\begin{align}
%\lambda^{k} \leftarrow \max\big(0,\kappa \big( E_{d}(T^{k-1}, U^{k-1}) -b \big) + \lambda^{k-1}\big).	
%\end{align}
%We simply use $\kappa = 1$ for all inputs throughout our paper, since it is appropriately small that would not overly change the Lagrangian.
\begin{align}
\lambda^{k} \leftarrow \max\big(0,\big( E_{d}(T^{k-1}, U^{k-1}) -b_d \big) + \lambda^{k-1}\big).	
\end{align}






%When distortion is a since it does not take into account the fact that we might start away from feasibility and want to iteratively improve both our primal variables $(T, U)$ and our dual variable $\lambda$. To smoothly update to a current $\lambda^{k}$ in iteration $k$ from a previous estimate $\lambda^{k-1}$, we add a simple quadratic regularizer $R(\lambda,\lambda^{k-1}) = \frac{1}{2\kappa} (\lambda- \lambda^{k-1})^2$ \justin{to what?} to make sure $\lambda$ iterates behave reasonably. \justin{Right now previous sentence sounds quite heuristic; can you cite an optimization algorithm that does this?} In practice, we simply set $\kappa = 1$.
%
%For iteration $k$ this gives us 
%\[ \min_{T,V} \max_{\lambda \geq 0} E_{s}(T^{k-1}) + \lambda \big( E_{d}(T^{k-1}, U^{k-1}) - b_d\big) - \frac{1}{2\kappa} (\lambda- \lambda^{k-1})^2 \]
%which can be solved in closed form as
%\[ \lambda^{k} \leftarrow \max\big(0,\kappa \big( E_{d}(T^{k-1}, U^{k-1}) -b \big) + \lambda^{k-1}\big) \]
%
%\danny{(Notice that throughout the above we can define a progressive $\lambda$ without needing to employ subgradients to reason about nonsmoothness in our sparsity energy.)}\justin{didn't follow this, not sure it's needed}



% \vova{The section below seem to be concerned with convergence of Alg 1, which is beyond self-weighted objective, IMHO... 
% It is also confusing that there a subsection 5.7 on convergence...}
%\subsection{Convergence}
%
%Our primal update converges when the UV map reaches the local minimizer of $L$ in both continuous and discrete searches. Similar to standard continuous search, our discrete topology search also converges at a local minimum, but in topology space, where changing to all neighboring topologies can not decrease $L$. Since we ensured monotonic decrease in $L$ over both the smooth and topology descent steps, we can prove that a near-stationary point with respect to both $T$ and $U$ can be reached for any input within a bounded number of alternations given a fixed lambda (Section~\ref{sec:convergence}).
%
%Theoretically, the dual update converges in two situations: either at $\lambda^* = 0$ when the primal minimizer $(T^*, U^*)$ is inside the feasible region ($E_d(T^*, U^*) < b_d$) with no seams, or at some $\lambda^* > 0$ with $(T^*, U^*)$ on the boundary of the feasible region ($E_d(T^*, U^*) = b_d$). However, for the latter case, since there is only a finite number of configurations of $T$, the near stationary $E_d$'s may not be exactly equal to $b_d$. Instead of setting a relatively large tolerance for detecting convergence, we stop when it first tries to violate distortion bound $b_d$ from being feasible. \justin{this was kind of vague; is it in the pseudocode?}
