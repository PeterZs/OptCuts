\documentclass[acmtog, review, anonymous]{acmart}

\usepackage{booktabs} % For formal tables

% TOG prefers author-name bib system with square brackets
\citestyle{acmauthoryear}
\setcitestyle{square}


\usepackage[ruled]{algorithm2e} % For algorithms
\renewcommand{\algorithmcfname}{ALGORITHM}
\SetAlFnt{\small}
\SetAlCapFnt{\small}
\SetAlCapNameFnt{\small}
\SetAlCapHSkip{0pt}
\IncMargin{-\parindent}

% Metadata Information
\acmJournal{TOG}
% \acmVolume{9}
% \acmNumber{4}
% \acmArticle{39}
% \acmYear{2010}
% \acmMonth{3}

% Copyright
%\setcopyright{acmcopyright}
%\setcopyright{acmlicensed}
%\setcopyright{rightsretained}
%\setcopyright{usgov}
% \setcopyright{usgovmixed}
%\setcopyright{cagov}
%\setcopyright{cagovmixed}

% DOI
% \acmDOI{0000001.0000001_2}

% Paper history
% \received{February 2007}
% \received{March 2009}
% \received[final version]{June 2009}
% \received[accepted]{July 2009}

% Document starts
\begin{document}
% Title portion
\title{OptCuts: Joint Optimization for Seam Placement and Parameterization of 3D Surfaces} 

\author{Minchen Li}
% \orcid{1234-5678-9012-3456}
% \affiliation{%
%   \institution{College of William and Mary}
%   \streetaddress{104 Jamestown Rd}
%   \city{Williamsburg}
%   \state{VA}
%   \postcode{23185}
%   \country{USA}}
\email{minchernl@gmail.com}
% \author{Valerie B\'eranger}
% \affiliation{%
%   \institution{Inria Paris-Rocquencourt}
%   \city{Rocquencourt}
%   \country{France}
% }
% \email{beranger@inria.fr}
% \author{Aparna Patel} 
% \affiliation{%
%  \institution{Rajiv Gandhi University}
%  \streetaddress{Rono-Hills}
%  \city{Doimukh} 
%  \state{Arunachal Pradesh}
%  \country{India}}
% \email{aprna_patel@rguhs.ac.in}
% \author{Huifen Chan}
% \affiliation{%
%   \institution{Tsinghua University}
%   \streetaddress{30 Shuangqing Rd}
%   \city{Haidian Qu} 
%   \state{Beijing Shi}
%   \country{China}
% }
% \email{chan0345@tsinghua.edu.cn}
% \author{Ting Yan}
% \affiliation{%
%   \institution{Eaton Innovation Center}
%   \city{Prague}
%   \country{Czech Republic}}
% \email{yanting02@gmail.com}
% \author{Tian He}
% \affiliation{%
%   \institution{University of Virginia}
%   \department{School of Engineering}
%   \city{Charlottesville}
%   \state{VA}
%   \postcode{22903}
%   \country{USA}
% }
% \affiliation{%
%   \institution{University of Minnesota}
%   \country{USA}}
% \email{tinghe@uva.edu}
% \author{Chengdu Huang}
% \author{John A. Stankovic}
% \author{Tarek F. Abdelzaher}
% \affiliation{%
%   \institution{University of Virginia}
%   \department{School of Engineering}
%   \city{Charlottesville}
%   \state{VA}
%   \postcode{22903}
%   \country{USA}
% }

\renewcommand\shortauthors{Li, M. et al}

\definecolor{gray}{rgb}{0.5,0.5,0.5}
\definecolor{green}{rgb}{0, 0.6, 0}
\definecolor{orange}{rgb}{1, 0.5, 0}
\definecolor{mahogany}{rgb}{0.75, 0.25, 0.0}
\definecolor{purple}{rgb}{0.6, 0, 0.6}
\definecolor{darkgreen}{rgb}{0, 0.3, 0}
\definecolor{orange}{rgb}{1, 0.5, 0.}
\newcommand{\minchen}[1]{\textcolor{blue}{\textbf{Minchen: #1}}}

\begin{abstract}
Parameterizing 3D surfaces to the 2D plane is a fundamental problem with applications in texture mapping, remeshing, and detail transfer, etc.
For most surfaces, one has to introduce discontinuities (seams) and distortion while producing a map. Existing techniques typically decide between the two independently: first placing seams and then minimizing distortion, and thus produce sub-optimal results.
We propose a joint discrete-continuous optimization framework that optimally and progressively introduce or remove seams (in topology steps) in between distortion minimizations (in descent steps). We use a linear combination of symmetric Dirichlet energy and seam length as objective, of which the stationary w.r.t. both UV topology and coordinates are guaranteed to be reached within a bounded number of alternating iterations per balancing factor, input model, and initial embedding. \minchen{[NOTE] (Here stationary w.r.t. UV topology is only in the approximation sense, because there might still be basic topological operations that could decrease the objective but end up not chosen because it's local evaluated energy decrease is not the largest one.)}

Specifically, in descent steps, we minimize symmetric Dirichlet energy using projected Newton method given the current UV topology. In topology steps, we search for a nearby UV topology that locally decrease the objective the most by querying a filtered set of basic topological operations. To be appropriately aggressive on searching in the topological space, \minchen{[TODO] we develop an analogous line search method as in continuous settings and allow multiple fracture initiation.}
Since in application scenarios, an upper bound for distortion or seam length is more intuitive than picking a balancing factor, we also provide a constrained optimization view of this broader problem that seeks stationary w.r.t. both primal (distortion and seams) and dual (balancing factor) variables subject to user specified distortion or seam length upper bounds.

Our method automatically produces high quality UV maps without any user assistance. \minchen{[TODO] We also show that given a UV configuration by other methods, our method can improve the distortion and seam placement}, and that our framework has the potential to handle bijectivity, seamlessness, and user preferences jointly within it as well.
\end{abstract}


%
% The code below should be generated by the tool at
% http://dl.acm.org/ccs.cfm
% Please copy and paste the code instead of the example below. 
%
\begin{CCSXML}
<ccs2012>
	<concept>
		<concept_id>10010147.10010371.10010396.10010398</concept_id>
		<concept_desc>Computing methodologies~Mesh geometry models</concept_desc>
		<concept_significance>500</concept_significance>
	</concept>
</ccs2012>
\end{CCSXML}

\ccsdesc[500]{Computing methodologies~Mesh geometry models}

%
% End generated code
%


\keywords{geometry processing, mesh parameterization, seam placement, numerical optimization, ...}



\maketitle

% !TeX root = OptCuts.tex

\section{Introduction}
% context
Mapping three-dimensional meshes to the plane is a fundamental task in computer graphics.  The two-dimensional mesh embeddings produced by mapping methods are commonly used to store reflectance functions, normals, and displacements
for the mesh, providing a domain for painting, synthesizing, and manipulating texture and geometric details. 
%Mesh parameterization is a critical task in computer graphics, with applications including texture mapping, remeshing, and detail transfer.  
%\vova{I removed non-texture apps for now, because I'm not sure if our formulation is as easy to motivate for remeshing and detail transfer.}
%
The usability of these embeddings is highly dependent on two interconnected factors: the surface distortion introduced by the mapping and the length of the surface cuts, forming seams across which the mapping is discontinuous~\cite{Hormann2008}. Both high distortion and longer seams are detrimental to downstream applications. Yet, reducing distortion below a desired 
%some %intrinsic surface %<--- what does this mean
bound typically requires introducing longer seams. 
% To do so, parameterization tools must cope with intertwined challenges in topology and geometry:  a high-quality parameterization must cut a surface into simple %disk-shape 
%patches so that each can be mapped to the plane with a reasonably small level of distortion.

Given its broad applicability, parameterization has long been a focus of research in geometry processing. Algorithms in this domain focus on these two key aspects of the problem \cite{Sheffer07_ParameterizationSurvey}.  Particularly well-studied are \emph{geometric} techniques that assume a surface has already been cut into disk-topology segments %charts %<--- charts are the maps
 which each then need to be mapped into the plane with minimal distortion while maintaining fixed connectivity; at this point, parameterization becomes a real-valued optimization problem that seeks to minimize changes in mesh angles and areas while maintaining local or global injectivity. Complementing these techniques, \emph{topological} algorithms find reasonable seams, either keeping the surface in one piece or partitioning it into individual segments that can then be parameterized with low distortion. %(Section~\ref{sec:related}).   

In contrast, we propose a joint optimization algorithm {\em OptCuts} that simultaneously optimizes for both seam length and the corresponding distortion of the embedding.
Our algorithm is based on a minimization model problem that directly and automatically balances between seam length and parametric distortion measures. Manually balancing distortion and seam quality requires a choice of a relative scaling factor between these two objectives. From a practical perspective, it is difficult for users to choose this factor as the two terms measure very different quantities and no such setting can provide a guarantee on the quality of the generated map's distortion. 
On the other hand, users typically have a clear sense of the amount of distortion they consider acceptable for their application. Motivated by this observation we cast our coupled seam and distortion optimization as a \emph{constrained} problem to find charts with locally minimal seam lengths that strictly satisfy a user-set distortion bound. Treating the distortion bound as a hard inequality constraint guarantees a pre-specified level of mapping quality, while enabling us to explore optimal seams that satisfy this bound. %In turn, o %<--- not sure what ``in turn'' means in this context

Prior methods coupling distortion reduction and cutting have generally required hand-tuning a number of user-exposed parameters and, as in the recently proposed AutoCuts~\cite{Poranne2017Autocuts}, even advocate manual intervention by interactively adjusting these parameters during the optimization process.

In contrast, OptCuts is a fully automatic optimization method: users provide their desired distortion bound and OptCuts then directly computes a parametrization satisfying this bound with locally minimal edge lengths. Maps provided are always locally injective and, as we will show, can additionally be constrained to be bijective and even support additional, user-provided seam placement constraints and biases when desired. 
%Likewise, as 
As demonstrated by our comparisons in Section~\ref{sec:results},
%, e.g.\ Figure~\ref{fig:autocut} 
%and ???~\danny{update when all comparison figures and tables are in.}, 
when compared to previous methods that do provide an automatic mode~\cite{BoundedDistortParam:2002,Poranne2017Autocuts}, OptCuts produces much shorter seams when its bound is set for the same achieved distortion. 
%Even in comparing to hand-tuned UV maps, we find that OptCuts consistently finds similar seam lengths for comparable or even better distortion bounds; see e.g., Figures ??? \danny{update when all comparison figures and tables are in.}. 
Likewise, as we show in Section~\ref{sec:var}, e.g. Figure \ref{fig:comp_Seamster}, OptCuts can also be used to polish any preexisting UV map irrespective of the method used to create it. OptCuts can take an arbitrary UV map as input and improve either seam length while preserving the current distortion bound or even improve upon distortion as well, by setting a lower distortion bound.

  To achieve these gains we begin by casting global parameterization as a constrained minimization, formulated with seam length as our objective and a distortion bound as our inequality constraint. We then observe that the \emph{Lagrangian} of this constrained minimization's saddle-point problem directly captures a multi-objective optimization formed by the weighted sum of our seam length measure and map distortion. However, the key observation here is that now there is a natural scaling implied between the two measures that is directly defined by the Lagrange multiplier of the distortion bound. Regularization of iterated updates to this multiplier then allow us to smoothly explore variations of the Lagrangian over the space of seam cuts. 
  
  Next, we observe that to solve this saddle-point problem we must optimize over both smooth vertex parameters and discrete changes in topology. Exhaustive search is clearly not an option. Instead, we propose a discrete-continuous optimization method that explores decrease of distortion and seam length over both classical, smooth descent directions and along propagations of topological merging and cutting operations on the UV mesh. When desired, we additionally enforce constraints to achieve globally bijective maps. Finally, we additionally allow UV artists the option to guide seam placement away from salient regions by enabling painting over the surface. OptCuts then avoids seam placement in the regions in proportion to the intensity of the painting.
  
  Together, these components form the core of our OptCuts algorithm. Over a wide range of examples we show that OptCuts efficiently achieves all attempted distortion bounds while locally minimizing seam length for both locally injective and bijective mappings. In Section~\ref{sec:results} we compare against both state of the art algorithms and industrial UV-parameterization tools and show that for the same achieved distortion bound, we consistently improve seam-length over prior automated methods, while our automated results closely match with the results of hand-tuned methods. We also evaluate OptCuts over a large benchmark of parametrization problems, demonstrating that across mesh scales and problem difficulties OptCuts successfully obtains user-specified distortion bounds while efficiently minimizing seam length. 
  
%% contribution
% contribution
\subsection{Contributions.}

%In summary we propose OptCuts, t%<---- of course it's a summary
To our knowledge, OptCuts is the first fully automated global parameterization algorithm that obtains bijective maps satisfying prescribed distortion bounds while locally minimizing seam length. To do so we first formulate a new, simple-to-state, constrained seam-length minimization model problem. We then solve our problem by our proposed discrete-continuous algorithm for the saddle-point problem using a combined discrete search over propagated mesh operations and smooth descent over vertex positions. We evaluate OptCuts to show efficient performance and scaling. Across a wide range of automated methods it improves over the state of the art, while automatically obtaining comparable quality results to hand-tuned parameterization methods. 


% !TeX root = OptCuts.tex

\section{Related Works}

related methods:\\
AutoCuts~\cite{Poranne2017Autocuts}\\
Seamster~\cite{Sheffer2002Seamster}\\
geometry images~\cite{Gu2002Geometry}\\
Multi-chart geometry images~\cite{Snyder2003Multi}\\
D-Chart~\cite{Julius2005D}\\
Boundary First Flattening~\cite{Sawhney:2017}\\
SeamCut~\cite{Lucquin:2017}\\
Bijective parameterization with free boundaries~\cite{Smith2015Bijective}\\
MIPS~\cite{Hormann2000MIPS}\\
ABF++~\cite{Sheffer2005ABFPP}
global parameterization methods?

%Seams, due to its discontinuous property, is not intuitive to be considered in traditional distortion minimization frameworks.
Due to discontinuities that occur when seams are introduced or removed, it is not intuitive to consider optimization of seam topology in the context of traditional frameworks for minimizing distortion during mesh parameterization.
%
\justin{couldn't follow this sentence (what does ``efficient'' or ``sparse'' mean in this context and what does it have to do with L2?):}\minchen{"efficient" and "sparse" means the number of seam edges should be in $O(\sqrt{numOfEdges})$. If one tries to model seam as a continuous energy (e.g.\ \cite{Poranne2017Autocuts}), residuals will be distributed evenly if the objective function is L2-type. To obtain sparsity structure in the result, less smooth energy such as L1-type energy is needed, which is similar to feature selection in statistical learning.}
Moreover, for seams to be efficient, it needs to be sparse, which is another challenge for optimizing it with L2-type distortion energies.


The recent AutoCuts~\cite{Poranne2017Autocuts} algorithm measures seam quality using an energy with discontinuities when triangles are glue together or disconnected.  Their procedure progressively builds up a parameterization starting from triangle soup, jointly improving topology and distortion via homotopy optimization. 
%
%We observed that i
While their method is among the first to optimize parameterization topology and geometry simultaneously, 
initially placing seams on all the edges introduces unneeded degrees of freedom and unnecessary computational expense:  Most of the triangles remain attached to their neighbors after their optimization procedure converges. Also, since their seam placement highly depends on the homotopy path, AutoCuts relies on user guidance to obtain good results, e.g.\ for parameter tuning, cut suggestion, and patch movement. %\danny{Here or elsewhere we should also add the observation that a full triangle soup initializer requires an awful lot of extra (and generally unnecessary) work to glue everything back together...}%\justin{how's the above?}

Our framework is different from well-known seam cutting algorithms like Geometry Images~\cite{Gu2002Geometry} and Seamster~\cite{Sheffer2002Seamster}, in which the core idea is to locate points of maximal currently predicted distortion and to add cut paths toward them. These heuristics do not perform well if no such obvious points exist, e.g.\ once distortion is distributed near-evenly across many surface points. Our framework in contrast searches for minimal cut elongation or shrinking steps that reduce a joint objective, and thus we expect it to be more efficient in such settings (Figure~\ref{cases where there are not many obvious extremal points}).

Although OptCuts does not require user assistance, it still allows users to communicate preferences on regional seam placement through edge weight painting (Figure~\ref{fig:edge_weight_painting}).
In addition, it can work with ``bespoke'' distortion energies when necessary. %our seams are optimal for the distortion energy used. 
For example, it creates different set of seams that benefit conformality if the objective function penalizes conformal distortion (Figure~\ref{results of our method with conformal distortion energy}).

\section{An Alternating Framework of Continuous and Discrete Optimization for Mesh Parameterization}

The most basic and intuitive mesh parameterization objective regarding both seams and distortion is minimizing distortion with as-sparse-as-possible seams introduced. However, seam sparsity usually leads to discontinuous energies w.r.t. UV coordinates $U \in \mathcal{R}^{2n_v}$, which is non-trivial to be considered into existing distortion minimization routines. Instead of progressively approximating seam sparsity energy with a continuous counterpart applying homotopy optimization method as~\cite{Poranne2017Autocuts}, we handle this discrete energy in a combinatoric way - searching in the topological space.

\subsection{Formulation}

This topological space is a directed graph $G_T$ with its vertices $v_T \in V_T$ being all possible UV topologies of a given 3D surface, and its edges $e_T \in E_T$ are the basic topological operations on a mesh such as vertex split, edge merge, etc, that can transform one UV topology to a nearby topology.

Now, if we consider both distortion and seam in one objective $E_w$, we can define the value $f_v$ of vertex $v_{T,i}$ as 
\[ f_v(v_{T,i}) = \min_{U} E_w \]
and the weights $f_w$ of edge $e_{T,m}$ from $v_{T,i}$ to $v_{T,j}$ could just be defined as 
\[ f_w(e_{T,m}) = f_v(v_{T,j}) - f_v(v_{T,i}) \]
Thus our problem could be written as
\[ \min_{U, v_T} E_w \]
which could be stated as to search for a $v_{T,i}$ on $G_T$ where all edges connected to it satisfies $f_w \geq 0$. 

However, computing $f_v$ for one UV topology requires a whole continuous optimization process, and even the number of neighbors of a UV topology is in the scale of $n_v^2$. Consequently, we construct a search path on $G_T$ by progressively introducing or removing seams, and we only estimate $f_w$ on a local stencil of $U$ for a filtered set of neighbors so that the whole process of continuous optimization is only conducted while necessary:

Let's consider a simple situation, minimizing symmetric Dirichlet energy~\cite{Smith2015Bijective}
\[ E_{SD} = \frac{1}{n_t \overline{|A|}} \sum_t |A_t|(\sigma_{t,1}^2 + \sigma_{t,2}^2 + \sigma_{t,1}^{-2} + \sigma_{t,2}^{-2}) \]
and total seam length
\[ E_{se} = \frac{1}{\sqrt{n_t}\overline{|e|}} \sum_{i \in \mathcal{S}} 2|e_i| \]
where a balancing factor $\lambda \in [0, 1]$ is controlling the ratio between the two: 
\[ E_w = \lambda E_{se} + (1 - \lambda) E_{SD} \]
We minimize $E_w$ by iteratively alternating between continuous optimization (in descent steps) and discrete optimization (in topology steps):
\begin{itemize}
\item In descent steps, we compute $f_v(v_{T,i})$ via projected Newton method~\cite{Teran2005Robust}:
\[ f_v(v_{T,i}) = E_{se,i} + \min_U E_{SD} \]
\item In topology steps, we estimate $f_v(v_{T,j})$ for a filtered set of neighbors on a local stencil of $U$ as $\hat{f}_v$ and move onto the neighbor $v_{T,i+1}$ with smallest $\hat{f}_v$.
\end{itemize}
If in a descent step, $f_v(v_{T,i}) \geq f_v(v_{T,i-1})$ is detected, we stop the process by rolling back to $v_{T,i-1}$, which is the stationary of $E_w$ w.r.t. both UV topology and UV coordinates that we are searching for.

\subsection{Convergence}

As our method is defined to guarantee convergence, we now analyze convergence rate. First, it's easy to see that $E_w$ is monotonically decreasing looking at each end of descent steps. Now we look at descent step $i$ and $i+1$, from $E^i_w \geq E^{i+1}_w$ we have
\[ E^i_{SD} - E^{i+1}_{SD} \geq \frac{\lambda}{1-\lambda} (E^{i+1}_{se} - E^i_{se}) \geq \frac{\lambda}{1-\lambda} \frac{1}{\sqrt{n_t}\overline{|e|}} 2|e|_{min} \]
if we now only consider splitting operations that keep increasing $E_{se}$. It's obvious that $E_{SD}$'s lower bound is defined to be $4$. So we have
\[ n_{alter} \leq \frac{(1-\lambda)\sqrt{n_t}\overline{|e|}}{2\lambda|e|_{min}} (E^0_{SD} - 4) \]
The most important hint we can read from this is, to accelerate convergence, we can move through multiple vertices on $G_T$ in each topology step to increase $E^{i+1}_{se} - E^i_{se}$. \textcolor{red}{Consequently, we build an anologous line search method to be appropriately agressive when searching in the topological space so that we won't fall into bad locally optimal UV topologies.}

\textcolor{red}{Merge operations should be defined carefully to ensure convergence, and the proof will need updates.}

\subsection{Potential Extensions}
It will be interesting to replace $E_{SD}$ with other types of distortion energies, especially conformal energies like MIPS~\cite{Hormann2000MIPS} to see how it behaves.
Besides, bijectivity could be potentially achieved by augmenting distortion energy with a penalty-based collision handling energy, possibly also assisted by air mesh method~\cite{?}.
Similarly, seamless properties could also be achieved by augmenting distortion energy with the correspondingly developed new differentiable objectives, and our alternating framework stays the same.

If an objective derived from an application is discontinuous and it could be expressed using mesh topology, then we can simply augment it into $E_{se}$ and tackle it in the topology steps. For example, the smoothness of seams, user preferences on regional seam placement, and properties related to charts should all be able to be considered in this way.

Besides, parallelism not only accelerate our topology steps, but also could improve the results by conducting an either broader or deeper search in the topological space.

\section{Descent Steps for Continuous Optimization}

\subsection{Newton-type Iterations}

for each descent step inner iteration $j$:

compute $E_{SD}$ Hessian proxy $P^j$ using projected Newton;

compute $E_{SD}$ gradient $g^j$;

solve for search direction $p^j$ ($P^j p^j = -g^j$) using PARDISO symmetric indefinite solver;

compute initial step size $\alpha^j_0$ by avoiding element inversion;

backtracking line search with Armijo rule;

update $U^{j+1} = U^j + \alpha^j p^j$;

record energy decrease $(1-\lambda_t)\Delta E_{SD}^j$;

\subsection{Potential Accelerations for Practical Use}

Since our topological operations only change the mesh locally both on connectivity and coordinates, we could also update the Hessian or the decomposition locally after topology changes to save time. Besides, it's also interesting to try other Hessian approximation methods like L-BFGS or Majorization to explore further acceleration by finding a balance between computational cost and convergence rate.

For convergence tolerance of descent steps, $||\nabla E_{SD}||^2 \leq 10^{-6}$ (note that our energy is normalized) works generally well for all input models judging from the initiated fracture in the following topology step. In fact more inexact solve performs well on most of the models with even $||\nabla E_{SD}||^2 \leq 10^{-4}$, but some may result even better with $||\nabla E_{SD}||^2 \leq 10^{-8}$. Since we are conducting non-convex optimization, $||\nabla E_{SD}||^2$ is not always decreasing, which is also why we don't use Wolfe conditions for line search. The argument here for tolerance issue is that, it depends on whether we are truly in the infinitesimal region of a stationary. Some configuration with $||\nabla E_{SD}||^2 \leq 10^{-6}$ may still not inside the infinitesimal region of a stationary, where if optimization goes on, the $||\nabla E_{SD}||^2$ will go up and then fall down again to a real stationary, which is understandable in non-convex optimization.

\section{Topology Steps for Discrete Optimization}

\subsection{Evaluating Topological Operations via Optimization on Local Stencils}

Candidate Filtering:
for each vertex
  compute divergence of local gradients
independently picking $\sqrt{n_{v,b}^i}$ boundary vertices and $\sqrt{n_{v,i}^i}$ interior vertices with largest divergence as candidates

Local Evaluation:
for each candidate vertex
  if on boundary
    for each interior incident edge
      split and compute $\Delta E_{SD,l}$ locally
      compute $\Delta E_{w,l} = (1 - \lambda_t) \Delta E_{SD,l} + \lambda_t \Delta E_{se}$
  else
    for each pair of incident edges forming a smooth path
      split and compute $\Delta E_{SD,l}$ locally
      compute $\Delta E_{w,l} = 0.5((1 - \lambda_t) \Delta E_{SD,l} + \lambda_t \Delta E_{se})$

split the vertex with largest $|\Delta E_{w,l}|$
turn on fracture propagation

\textcolor{red}{try larger stencils}

\textcolor{red}{enable merge operation}

\textcolor{red}{\subsection{Line Search in Topological Space}}

Current Fracture Propagation:
if fracture propagation is on
  for each fracture tail vertex $k$
    for each interior incident edge of $k$
      split and compute $\Delta E_{SD, l}$ locally
      compute $\Delta E_{w,l} = (1 - \lambda_t) \Delta E_{SD,l} + \lambda_t \Delta E_{se}$
  if the largest $|\Delta E_{w,l}|$ is larger than $|(1-\lambda_t)\Delta E_{SD}^j|$
    propagate fracture by splitting the vertex
  else
    turn off fracture propagation for the rest of the current descent step

% !TeX root = OptCuts.tex

\section{Self-Weighted Objective}
\label{sec:self_weighting}
\danny{This section gives the ``outer'' loop details - I'd suggest moving it above the sections on the inner-loop details in the paper as it's short and would help readers understand the algorithm structure (roadmap!). Then jump into the inner loop detail sections - this would follow the structure of the actual algorithm.}
\subsection{Formulation}
At each inner iterate $k+1$, we fix some $\lambda^{k+1}$ and minimize the bi-objective 
\[ \min_{T,V} E_{SE}(V,T) + \lambda^{k+1} E_{SD}(V,T) \]

How do we get $\lambda^{k+1}$? \justin{not sure a rhetorical question is needed}

Our overall minimization is inequality constrained with a specified upper bound $b \in \mathbb{R}_+$ on distortion. \justin{I moved a parenthetical to a Minchen comment assuming he'll write it more formally}\minchen{(L2 norm on SD  energy for now - pretty easy to modify to an extremal measure if we want later on.)}

Our model problem minimization is then 
\[ \min_{T,V} E_{SE}(V,T) :  b - E_{SD}(V,T) \geq 0 \]

Or, equivalently,
\[ \min_{T,V} \max_{\lambda \geq 0} E_{SE}(V,T) + \lambda \big( E_{SD}(V,T) - b\big) \]
\justin{the two equations above are repeats from sec3; remove them and just use eq numbers.}

The inner objective is nonsmooth in $\lambda$ since it does not take into account %very nicely 
the fact that per-iteration we will start away from feasibility and want to iteratively improve both our primal variables $\{V,T\}$ and our dual variable $\lambda$. \justin{I didn't follow the previous sentence} To smoothly update to a current $\lambda^{k+1}$ from a previous estimate $\lambda^k$ we will add a regularizer $R(\lambda,\lambda^k)$ to make sure $\lambda$ iterates behave reasonably. For now lets stick with something simple:  a quadratic regularizer should do the trick  $R =\frac{1}{2\kappa} (\lambda- \lambda^k)^2$.  \justin{Previous sentence is too informal, and not sure ``for now'' is right}

For iteration $k+1$ this gives us 
\[ \min_{T,V} \max_{\lambda \geq 0} E_{SE}(V,T) + \lambda \big( E_{SD}(V,T) - b\big) - \frac{1}{2\kappa} (\lambda- \lambda^k)^2 \]

And now we can first solve closed form for $\lambda$ as 
\[ \lambda^{k+1} \leftarrow argmax_{\lambda \geq 0} E_{SE}(V,T) + \lambda \big( E_{SD}(V,T) - b\big) - \frac{1}{2\kappa} (\lambda- \lambda^k)^2 \]
giving us 
\[ \lambda^{k+1} \leftarrow \max\big(0,\kappa \big( E_{SD}(V,T) -b \big) + \lambda^k\big) \]
We then can solve the inner iteration (with both discrete topology steps and smooth steps) with the energy 
\[ \min_{T,V}  E_{SE}(V,T) + \lambda^{k+1}  E_{SD}(V,T), \]
followed by the next update of dual variable $\lambda$.

(Notice that throughout the above we can define a progressive $\lambda$ without needing to employ subgradients to reason about nonsmoothness in our sparsity energy.)\justin{didn't follow this, not sure it's needed}

\justin{Might be worth adding a few sentences referencing that this is an instance of previous optimization techniques.  Can you call the lambda update a proximal step?}

\subsection{Implementation}

\begin{algorithm}[!h]
\SetAlgoLined
\KwData{Input model with initial UV map, expected $E_{SD}$ upper bound $b$}
\KwResult{UV map with $argmin_{v_T} E_{se}$ topology and $E_{SD} <= b$}

$\lambda \leftarrow +\infty$ \\
\For{each alternating iteration $i$}{
  descent step $i$\;
  \If{$E^i_{SD} <= b$}{
  	$j \leftarrow i$
    break;
  }
  $i \leftarrow i + 1$\;
  topology step $i$\;
}

$\lambda \leftarrow (E^j_{se} - E^{j-1}_{se})/(E^{j-1}_{SD}-E^{j}_{SD})$\;

\For{each alternating iteration $i$}{
  topology step\;
  descent step\;

  \If{$(E^i_{se}, E^i_{SD}, \lambda)$ has occurred before}{
    set to best feasible UV map and break\;
  }
  \If{at $\min_{V,T} E_{se} + \lambda E_{SD}$}{
    \If{$E^i_{SD} <= b$  $E^i_{SD} >= b - tol$}{
      set to best feasible UV map and break\;
    }
  }
  
  $\lambda \leftarrow \max\big(0,\kappa(E_{SD}(V,T) -b) + \lambda \big)$\;
  \If{at $\min_{V,T} E_{se} + \lambda E_{SD}$}{
    \If{picking the wrong type of operation}{
      do
        $\lambda \leftarrow \max\big(0,\kappa(E_{SD}(V,T) -b) + \lambda \big)$\;
      while picking the wrong type of operation
    }
    \Else {
      while candidate selection didn't change
        $\lambda \leftarrow \max\big(0,\kappa(E_{SD}(V,T) -b) + \lambda \big)$\;
      end
    }
  }
}

\caption{Self-Weighted $E_w$}
\end{algorithm}

\minchen{[NOTE] To understand alternating lambda updates between PN iterations: lambda update changes the energy manifold, so it couldn't be too frequent or the manifold can't really be explored enough, nor it could be too seldom or it would be trapped in a local region.}

% !TeX root = OptCuts.tex

\section{Results and Discussion}
\label{sec:results}

\subsection{Experiments}

\paragraph{Initial Embedding} For disk-topology surfaces, we map the longest boundary to a circle preserving edge length and obtain a Tutte embedding with uniform weights \minchen{[TODO] see whether using MVC weights converge faster}. For closed surfaces (including high-genus surfaces), we first apply some simple heuristics to obtain an initial seam, and then treat them as disk-topology surfaces. 
The heuristics for genus-0 surfaces include farthest point cut and random/curvest one point cut \minchen{[TODO] decide one and add description, or use all?}; for high-genus surfaces, we follow Crane et al.~\shortcite{Crane:2013:DGP} to detect homology generators and then cut along all of them \minchen{[TODO]}.
In order to show that we search for locally optimal UV maps regardless of the given initial embedding, we run our method starting from triangle soup and preliminary UV maps produced by other methods or by the users. The output maps are still with high quality (Figure~\ref{fig:bad_init_still_ends_well}) \minchen{[TODO]}.

\paragraph{Quality and Timing Comparisons} \minchen{[TODO], also provide detailed settings on the compared methods, and how much user assistance was needed for other methods?} We demonstrate our framework's capabilities by first comparing to AutoCuts~\cite{Poranne2017Autocuts} and two typical classic seam cutting methods~\cite{Gu2002Geometry,Sheffer2002Seamster}. Given the same input surface and initial UV map, we efficiently reach identical distortion bounds with shorter seam lengths (Figure~\ref{fig:QT_comp}). 
When we change the settings in order to obtain nearly isometric UV maps, the quality of the seams by other methods drop drastically while our method keeps generating high-quality seams (Figure~\ref{fig:strict_bounds_comp}).

\paragraph{Triangulation Invariance} \minchen{[TODO] not sure whether applicable}

\paragraph{Scalability Test} \minchen{[TODO]}

\subsection{Variations}

Without changing the framework, simply reformulating $E_w = E_s + \lambda E_d$ according to different needs enables OptCuts to solve mesh parameterization problems in many variations:

\paragraph{Global Bijectivity} \minchen{[DOING]} Augmenting our $E_s$ with a collision handling energy $E_b$ will easily achieve joint seam placement and bijective mesh parameterization. We show that by adding a scaffold mesh~\cite{Jiang2017Simplicial} to the voided regions of the UV map and preventing the scaffold mesh from degenerate, our method automatically generate high-quality bijective maps with optimal seams different from that of locally injective parameterization (Figure~\ref{fig:bijective_vs_injective}).

\paragraph{Conformal Parameterization} \minchen{[TODO]} Using a conformal energy~\cite{Hormann2000MIPS,Sheffer2005ABFPP} for $E_s$ will achieve joint seam placement and conformal parameterization. Figure~\ref{fig:conformal_vs_isometry} shows some results with $E_s = E_{ABForMIPS}$~\cite{} compared to results with $E_s = E_{SD}$, where different seams are generated while our framework stays the same.

\paragraph{Regional Seam Placement} \minchen{[TODO]} On the discrete side, if we reweight $E_{SL}$ with an edge prior provided by the user or an algorithm~\cite{} as
\[ E_s = \hat{E}_{SL} = \sum_{i\in\mathcal{S}} w_{SL,i} E_{SL,i} \quad w_{SL,i} \in \mathcal{R^+} \]
we could bias the seam placement towards regions e.g. where continuity is less in demand (Figure~\ref{fig:regional_seam_placement}).

discoveries? like interior splits?

\section{Conclusions and Future Works}

\begin{itemize}
\item Our method doesn't provide globally optimal solutions, the results are still locally optimal, but w.r.t. both seams and distortion, which is better than previous 2-pass methods that breaks the correlation between seams and distortion.
\item take advantage of basic SIMD type of parallelism for accelerating query and improving results' quality by directly evaluating $f_v$ for neighbors and track multiple branches, very useful for practical implementations
\item if the user won't mind getting a slightly different triangulation, we could also create fractures in the interior of an element and locally remesh the stencil
\item start and solve in 3D by reducing curvature so that the need for locally injective initial embedding in parameterization problems could be eliminated, and the result is only "biased" by it's 3D shape, which is the most reasonable bias
\item try conformal energy like MIPS
\item \textcolor{red}{bijectivity}, seamless, and other augmentation of continuous energy?
\item handle user preferences on seam placement
\item seam smoothness, patch related discrete energy augmentation?
\end{itemize}

\section{Acknowledgements}

\bibliographystyle{ACM-Reference-Format}
\bibliography{OptCuts} 

\end{document}
