\documentclass[acmtog, review, anonymous]{acmart}

\usepackage{booktabs} % For formal tables

% TOG prefers author-name bib system with square brackets
\citestyle{acmauthoryear}
\setcitestyle{square}


\usepackage[ruled]{algorithm2e} % For algorithms
\renewcommand{\algorithmcfname}{ALGORITHM}
\SetAlFnt{\small}
\SetAlCapFnt{\small}
\SetAlCapNameFnt{\small}
\SetAlCapHSkip{0pt}
\IncMargin{-\parindent}
\SetKwRepeat{Do}{do}{while}

\usepackage{amsmath}
\DeclareMathOperator*{\argmax}{arg\,max}
\DeclareMathOperator*{\argmin}{arg\,min}
 
% Metadata Information
\acmJournal{TOG}
% \acmVolume{9}
% \acmNumber{4}
% \acmArticle{39}
% \acmYear{2010}
% \acmMonth{3}

% Copyright
%\setcopyright{acmcopyright}
%\setcopyright{acmlicensed}
%\setcopyright{rightsretained}
%\setcopyright{usgov}
% \setcopyright{usgovmixed}
%\setcopyright{cagov}
%\setcopyright{cagovmixed}

% DOI
% \acmDOI{0000001.0000001_2}

% Paper history
% \received{February 2007}
% \received{March 2009}
% \received[final version]{June 2009}
% \received[accepted]{July 2009}

% Document starts
\begin{document}
% Title portion
\title{OptCuts: Joint %Optimization for 
Seam Placement and Parameterization for 3D Surfaces} 

\author{Minchen Li}
% \orcid{1234-5678-9012-3456}
% \affiliation{%
%   \institution{College of William and Mary}
%   \streetaddress{104 Jamestown Rd}
%   \city{Williamsburg}
%   \state{VA}
%   \postcode{23185}
%   \country{USA}}
\email{minchernl@gmail.com}
% \author{Valerie B\'eranger}
% \affiliation{%
%   \institution{Inria Paris-Rocquencourt}
%   \city{Rocquencourt}
%   \country{France}
% }
% \email{beranger@inria.fr}
% \author{Aparna Patel} 
% \affiliation{%
%  \institution{Rajiv Gandhi University}
%  \streetaddress{Rono-Hills}
%  \city{Doimukh} 
%  \state{Arunachal Pradesh}
%  \country{India}}
% \email{aprna_patel@rguhs.ac.in}
% \author{Huifen Chan}
% \affiliation{%
%   \institution{Tsinghua University}
%   \streetaddress{30 Shuangqing Rd}
%   \city{Haidian Qu} 
%   \state{Beijing Shi}
%   \country{China}
% }
% \email{chan0345@tsinghua.edu.cn}
% \author{Ting Yan}
% \affiliation{%
%   \institution{Eaton Innovation Center}
%   \city{Prague}
%   \country{Czech Republic}}
% \email{yanting02@gmail.com}
% \author{Tian He}
% \affiliation{%
%   \institution{University of Virginia}
%   \department{School of Engineering}
%   \city{Charlottesville}
%   \state{VA}
%   \postcode{22903}
%   \country{USA}
% }
% \affiliation{%
%   \institution{University of Minnesota}
%   \country{USA}}
% \email{tinghe@uva.edu}
% \author{Chengdu Huang}
% \author{John A. Stankovic}
% \author{Tarek F. Abdelzaher}
% \affiliation{%
%   \institution{University of Virginia}
%   \department{School of Engineering}
%   \city{Charlottesville}
%   \state{VA}
%   \postcode{22903}
%   \country{USA}
% }

\renewcommand\shortauthors{Li, M. et al}

\definecolor{gray}{rgb}{0.5,0.5,0.5}
\definecolor{green}{rgb}{0, 0.6, 0}
\definecolor{orange}{rgb}{1, 0.5, 0}
\definecolor{mahogany}{rgb}{0.75, 0.25, 0.0}
\definecolor{purple}{rgb}{0.6, 0.1, 0.6}
\definecolor{darkgreen}{rgb}{0, 0.3, 0}
\definecolor{orange}{rgb}{1, 0.5, 0.}
\newcommand{\minchen}[1]{\textcolor{blue}{\textbf{Minchen: #1}}}
\newcommand{\justin}[1]{\textcolor{red}{\textbf{Justin: #1}}}
\newcommand{\alla}[1]{\textcolor{orange}{\textbf{Alla: #1}}}
\newcommand{\danny}[1]{\textcolor{purple}{\textbf{Danny: #1}}}
\newcommand{\vova}[1]{\textcolor{green}{\textbf{Vova: #1}}}

\begin{abstract}
% context and need
Mapping 3D surfaces to the plane is a fundamental geometry processing problem. For most surfaces, one has to introduce discontinuities (seams) and distortion to produce a valid map. Most existing techniques largely consider seam placement and distortion minimization independently, first placing seams and then minimizing distortion; this leads to suboptimal output. Some recent attempts to address this issue ask the user to choose a potentially counterintuitive parameter weighting between seam quality and distortion and can require manual intervention to obtain a quality result. 
\vova{Above does not encapsulate geometry images and olga's greedy seaming approach. At the same time I don't want to over-crowd abstract with these details. 
Maybe we skip the discussion of previous work here simply motivate our approach? ``A 3D artist typically aims at a UV map that does not exceed certain distortion level (which might lead to stretching artifacts), while minimizing the seam length (which lead to discontinuities). Motivated by this observation, we propose a novel problem formulation and a joint discrete continuous optimization method...''} As an alternative, we propose {\em OptCuts}, a discrete--continuous optimization method that automatically optimizes seam length while satisfying user-specified distortion bounds. Starting with an initial embedding, OptCuts alternates between topology descent steps and smooth descent steps to search for an optimal UV map.  We demonstrate that OptCuts automatically produces high-quality, globally bijective UV maps without user intervention and can generate seams whose structure is informed by a range of standard distortion measures as well as user preferences on regional seam placement.
We also demonstrate extensions in our optimization framework enabling global bijectivity and incorporating user guidance. 
%
\vova{Removed: ``seamlessness, and other discrete-continuous geometry processing problems''. AFAIK we are not planning to show it, correct?}
\end{abstract}


%
% The code below should be generated by the tool at
% http://dl.acm.org/ccs.cfm
% Please copy and paste the code instead of the example below. 
%
\begin{CCSXML}
<ccs2012>
	<concept>
		<concept_id>10010147.10010371.10010396.10010398</concept_id>
		<concept_desc>Computing methodologies~Mesh geometry models</concept_desc>
		<concept_significance>500</concept_significance>
	</concept>
</ccs2012>
\end{CCSXML}

\ccsdesc[500]{Computing methodologies~Mesh geometry models}

%
% End generated code
%


\keywords{geometry processing, mesh parameterization, seam placement, numerical optimization, ...}



\maketitle

% !TeX root = OptCuts.tex

\section{Introduction}
% context
Mapping three-dimensional meshes to the plane is a fundamental task in computer graphics.  The two-dimensional mesh embeddings produced by mapping methods are commonly used to store reflectance functions, normals, and displacements
for the mesh, providing a domain for painting, synthesizing, and manipulating texture and geometric details. 
%Mesh parameterization is a critical task in computer graphics, with applications including texture mapping, remeshing, and detail transfer.  
%\vova{I removed non-texture apps for now, because I'm not sure if our formulation is as easy to motivate for remeshing and detail transfer.}
%
The usability of these embeddings is highly dependent on two interconnected factors: the surface distortion introduced by the mapping and the length of the surface cuts, forming seams across which the mapping is discontinuous~\cite{Hormann2008}. Both high distortion and longer seams are detrimental to downstream applications. Yet, reducing distortion below a desired 
%some %intrinsic surface %<--- what does this mean
bound typically requires introducing longer seams. 
% To do so, parameterization tools must cope with intertwined challenges in topology and geometry:  a high-quality parameterization must cut a surface into simple %disk-shape 
%patches so that each can be mapped to the plane with a reasonably small level of distortion.

Given its broad applicability, parameterization has long been a focus of research in geometry processing. Algorithms in this domain focus on these two key aspects of the problem \cite{Sheffer07_ParameterizationSurvey}.  Particularly well-studied are \emph{geometric} techniques that assume a surface has already been cut into disk-topology segments %charts %<--- charts are the maps
 which each then need to be mapped into the plane with minimal distortion while maintaining fixed connectivity; at this point, parameterization becomes a real-valued optimization problem that seeks to minimize changes in mesh angles and areas while maintaining local or global injectivity. Complementing these techniques, \emph{topological} algorithms find reasonable seams, either keeping the surface in one piece or partitioning it into individual segments that can then be parameterized with low distortion. %(Section~\ref{sec:related}).   

In contrast, we propose a joint optimization algorithm {\em OptCuts} that simultaneously optimizes for both seam length and the corresponding distortion of the embedding.
Our algorithm is based on a minimization model problem that directly and automatically balances between seam length and parametric distortion measures. Manually balancing distortion and seam quality requires a choice of a relative scaling factor between these two objectives. From a practical perspective, it is difficult for users to choose this factor as the two terms measure very different quantities and no such setting can provide a guarantee on the quality of the generated map's distortion. 
On the other hand, users typically have a clear sense of the amount of distortion they consider acceptable for their application. Motivated by this observation we cast our coupled seam and distortion optimization as a \emph{constrained} problem to find charts with locally minimal seam lengths that strictly satisfy a user-set distortion bound. Treating the distortion bound as a hard inequality constraint guarantees a pre-specified level of mapping quality, while enabling us to explore optimal seams that satisfy this bound. %In turn, o %<--- not sure what ``in turn'' means in this context

Prior methods coupling distortion reduction and cutting have generally required hand-tuning a number of user-exposed parameters and, as in the recently proposed AutoCuts~\cite{Poranne2017Autocuts}, even advocate manual intervention by interactively adjusting these parameters during the optimization process.

In contrast, OptCuts is a fully automatic optimization method: users provide their desired distortion bound and OptCuts then directly computes a parametrization satisfying this bound with locally minimal edge lengths. Maps provided are always locally injective and, as we will show, can additionally be constrained to be bijective and even support additional, user-provided seam placement constraints and biases when desired. 
%Likewise, as 
As demonstrated by our comparisons in Section~\ref{sec:results},
%, e.g.\ Figure~\ref{fig:autocut} 
%and ???~\danny{update when all comparison figures and tables are in.}, 
when compared to previous methods that do provide an automatic mode~\cite{BoundedDistortParam:2002,Poranne2017Autocuts}, OptCuts produces much shorter seams when its bound is set for the same achieved distortion. 
%Even in comparing to hand-tuned UV maps, we find that OptCuts consistently finds similar seam lengths for comparable or even better distortion bounds; see e.g., Figures ??? \danny{update when all comparison figures and tables are in.}. 
Likewise, as we show in Section~\ref{sec:var}, e.g. Figure \ref{fig:comp_Seamster}, OptCuts can also be used to polish any preexisting UV map irrespective of the method used to create it. OptCuts can take an arbitrary UV map as input and improve either seam length while preserving the current distortion bound or even improve upon distortion as well, by setting a lower distortion bound.

  To achieve these gains we begin by casting global parameterization as a constrained minimization, formulated with seam length as our objective and a distortion bound as our inequality constraint. We then observe that the \emph{Lagrangian} of this constrained minimization's saddle-point problem directly captures a multi-objective optimization formed by the weighted sum of our seam length measure and map distortion. However, the key observation here is that now there is a natural scaling implied between the two measures that is directly defined by the Lagrange multiplier of the distortion bound. Regularization of iterated updates to this multiplier then allow us to smoothly explore variations of the Lagrangian over the space of seam cuts. 
  
  Next, we observe that to solve this saddle-point problem we must optimize over both smooth vertex parameters and discrete changes in topology. Exhaustive search is clearly not an option. Instead, we propose a discrete-continuous optimization method that explores decrease of distortion and seam length over both classical, smooth descent directions and along propagations of topological merging and cutting operations on the UV mesh. When desired, we additionally enforce constraints to achieve globally bijective maps. Finally, we additionally allow UV artists the option to guide seam placement away from salient regions by enabling painting over the surface. OptCuts then avoids seam placement in the regions in proportion to the intensity of the painting.
  
  Together, these components form the core of our OptCuts algorithm. Over a wide range of examples we show that OptCuts efficiently achieves all attempted distortion bounds while locally minimizing seam length for both locally injective and bijective mappings. In Section~\ref{sec:results} we compare against both state of the art algorithms and industrial UV-parameterization tools and show that for the same achieved distortion bound, we consistently improve seam-length over prior automated methods, while our automated results closely match with the results of hand-tuned methods. We also evaluate OptCuts over a large benchmark of parametrization problems, demonstrating that across mesh scales and problem difficulties OptCuts successfully obtains user-specified distortion bounds while efficiently minimizing seam length. 
  
%% contribution
% contribution
\subsection{Contributions.}

%In summary we propose OptCuts, t%<---- of course it's a summary
To our knowledge, OptCuts is the first fully automated global parameterization algorithm that obtains bijective maps satisfying prescribed distortion bounds while locally minimizing seam length. To do so we first formulate a new, simple-to-state, constrained seam-length minimization model problem. We then solve our problem by our proposed discrete-continuous algorithm for the saddle-point problem using a combined discrete search over propagated mesh operations and smooth descent over vertex positions. We evaluate OptCuts to show efficient performance and scaling. Across a wide range of automated methods it improves over the state of the art, while automatically obtaining comparable quality results to hand-tuned parameterization methods. 


% !TeX root = OptCuts.tex

\section{Related Works}

related methods:\\
AutoCuts~\cite{Poranne2017Autocuts}\\
Seamster~\cite{Sheffer2002Seamster}\\
geometry images~\cite{Gu2002Geometry}\\
Multi-chart geometry images~\cite{Snyder2003Multi}\\
D-Chart~\cite{Julius2005D}\\
Boundary First Flattening~\cite{Sawhney:2017}\\
SeamCut~\cite{Lucquin:2017}\\
Bijective parameterization with free boundaries~\cite{Smith2015Bijective}\\
MIPS~\cite{Hormann2000MIPS}\\
ABF++~\cite{Sheffer2005ABFPP}
global parameterization methods?

%Seams, due to its discontinuous property, is not intuitive to be considered in traditional distortion minimization frameworks.
Due to discontinuities that occur when seams are introduced or removed, it is not intuitive to consider optimization of seam topology in the context of traditional frameworks for minimizing distortion during mesh parameterization.
%
\justin{couldn't follow this sentence (what does ``efficient'' or ``sparse'' mean in this context and what does it have to do with L2?):}\minchen{"efficient" and "sparse" means the number of seam edges should be in $O(\sqrt{numOfEdges})$. If one tries to model seam as a continuous energy (e.g.\ \cite{Poranne2017Autocuts}), residuals will be distributed evenly if the objective function is L2-type. To obtain sparsity structure in the result, less smooth energy such as L1-type energy is needed, which is similar to feature selection in statistical learning.}
Moreover, for seams to be efficient, it needs to be sparse, which is another challenge for optimizing it with L2-type distortion energies.


The recent AutoCuts~\cite{Poranne2017Autocuts} algorithm measures seam quality using an energy with discontinuities when triangles are glue together or disconnected.  Their procedure progressively builds up a parameterization starting from triangle soup, jointly improving topology and distortion via homotopy optimization. 
%
%We observed that i
While their method is among the first to optimize parameterization topology and geometry simultaneously, 
initially placing seams on all the edges introduces unneeded degrees of freedom and unnecessary computational expense:  Most of the triangles remain attached to their neighbors after their optimization procedure converges. Also, since their seam placement highly depends on the homotopy path, AutoCuts relies on user guidance to obtain good results, e.g.\ for parameter tuning, cut suggestion, and patch movement. %\danny{Here or elsewhere we should also add the observation that a full triangle soup initializer requires an awful lot of extra (and generally unnecessary) work to glue everything back together...}%\justin{how's the above?}

Our framework is different from well-known seam cutting algorithms like Geometry Images~\cite{Gu2002Geometry} and Seamster~\cite{Sheffer2002Seamster}, in which the core idea is to locate points of maximal currently predicted distortion and to add cut paths toward them. These heuristics do not perform well if no such obvious points exist, e.g.\ once distortion is distributed near-evenly across many surface points. Our framework in contrast searches for minimal cut elongation or shrinking steps that reduce a joint objective, and thus we expect it to be more efficient in such settings (Figure~\ref{cases where there are not many obvious extremal points}).

Although OptCuts does not require user assistance, it still allows users to communicate preferences on regional seam placement through edge weight painting (Figure~\ref{fig:edge_weight_painting}).
In addition, it can work with ``bespoke'' distortion energies when necessary. %our seams are optimal for the distortion energy used. 
For example, it creates different set of seams that benefit conformality if the objective function penalizes conformal distortion (Figure~\ref{results of our method with conformal distortion energy}).

% !TeX root = OptCuts.tex

\section{Problem Statement}
%\minchen{As Vova suggested, describe Self-Weighting as our main formulation before Joint Discrete-Continuous Search, so does Abstract, Introduction, and Section 4 and 5.}
% \danny{Also took a pretty heavy pass at re-organizing this section and the next.}

Given an input triangle mesh $M=(V,F)$ of a three-dimensional surface with vertices $V$, and faces $F$, we seek a UV map with topology $T^*=(V_{T^*}, F_{T^*})$ and a corresponding two-dimensional embedding of vertex coordinates, $U^* \in \mathbb{R}^{2 |V_{T^*}|}$, that locally minimize the constrained parametrization problem
%
\begin{align}
	\min_{T,U} E_s(T) \quad s.t. \quad E_d(T,U) \leq b_d\ \ \text{and}\ \ (T, U) \in \mathcal{I}.
	\label{eq:p1}
\end{align}
%\justin{Might be worth mentioning in a sentence what your representation of $T$ is} \minchen{I haven't thought about defining $T$ explicitly. If needed, it can be a vector of boolean variables representing whether each edge on the 3D surface is a seam edge, subject to manifold mesh constraints.}
%\vova{Binary value over edges requires additional post-process to identify which of two vertices are duplicated and which triangles they should be assigned to. 
%Why not just define a new mesh $(V_T, F_T)$ and a bijective map $f_T: F \rightarrow F_T$?}\danny{My suggestion is that we stick as closely as possible to the rep used in Minchen's code - I've asked him what he uses - we can monge that into text once he replies.} \minchen{Similar to Vova said, I simply use the vertex indices of each face to represent topology as what .obj file does. Should we mention it here or in the implementation section?}
%
Here $V_{T^*}$ forms a superset of $V$ with possibly duplicated vertices, and $F_{T^*}$ is the set of faces indexed into this new set of vertices. We use $\mathcal{I}$ to define the set of \emph{either} locally injective \emph{or} globally bijective UV maps; in the following we will first initially focus on the locally injective set and then later cover our extension to bijectivity. Energies $E_s$ and $E_d$ respectively measure seam quality and map distortion while $b_d$ is a user-specified \emph{upper bound} on the acceptable distortion of the generated map. This optimization is always feasible as in the limit more cuts will always satisfy any distortion bound. %\danny{If we truly want to make bijectivity a first-class component of our method (I'd suggest we do then we should add the bijectivity constraint to the optimization problem statement above and also the derivation that follows - happy to do this once we decide.}

In general, distortion measures are smooth albeit \emph{nonconvex}, while seam measures are \emph{nonsmooth} as they take a discrete jump when mesh edges are cut or merged. Hence, we must address both challenges to design a suitable optimization method. 

%
% e.g. symmetric Dirichlet, MIPS, etc, to measure distortion over the mapped domain. In what follows, for simplicity we largely focus on the symmetric Dirichlet energy and defer examples with other distortion measures for our results to Section\ \ref{}.

%Given an input triangle mesh $M$ of a 3D surface with $n_p$ vertices and its initial UV map with topology $T^0$ and coordinates $U^0 \in \mathbb{R}^{2n_p}$, we automatically search for
%\[ \argmin_{T,U} L(T,U) \]
%where $L(T,U) = E_s(T) + \lambda E_d(T,U)$ is the total energy combining both seam energy $E_s$ and distortion energy $E_d$ with a balancing factor $\lambda \in \mathbb{R^+}$. Notice that usually $E_d$ is smooth, while $E_s$ is defined to be non-differentiable, which gives us a discrete-continuous problem.

\subsection{Dual Objective}
We construct the Lagrangian for (\ref{eq:p1})  % no need to cite the definition of a Lagrangian
\begin{align}
	L(T,U,\lambda) = E_s(T) + \lambda(E_d(T,U) - b_d),
	\label{eq:L}
\end{align}
%
to form the equivalent saddle-point problem~\cite{Bertsekas:2016:NOP} defined over primal variables $T,U$ and dual variable $\lambda$:
%
\begin{align}
	\min_{T,U} \max_{\lambda\geq0} L(T,U,\lambda).
	\label{eq:p2}
\end{align}
%
Here $\lambda \in \mathbb{R_+}$ is the Lagrange multiplier for our distortion bound. On examination the Lagrangian $L$ can be seen as a multi-objective balancing between distortion and seam quality as dictated by $\lambda$. Here, however, $\lambda$ effectively applies a local scaling between the seam and distortion terms that is implied %directly and automatically given 
by the user-specified distortion bound. Within iterations of OptCuts, our algorithm to solve \eqref{eq:p1}, $\lambda$ will thus grow as we threaten to violate our distortion bound and so prioritize distortion minimization; similarly, $\lambda$ will decrease toward $0$ as our bound is strictly satisfied to prioritize seam quality.


\subsection{Mapping Quality Measures}
Concretely we formulate our seam-quality measure as the normalized total seam length
\begin{align}
E_s 
%= E_{SL} 
= \frac{1}{\sqrt{(\sum_{t\in\mathcal{F}} |A_t|)/\pi}} \sum_{i \in \mathcal{S}} |e_i|
\end{align}
where $\mathcal{S}$ is the set of all seam edges on the input surface and $|e_i|$ is the length of edge $i$ in the input mesh.
To measure distortion over the mapped domain we apply the symmetric Dirichlet energy~\cite{Smith2015Bijective} normalized by surface area~\footnote{For simplicity we focus on symmetric Dirichlet here; alternate distortion energies follow similarly.}, 
\begin{align} 
E_d 
%= E_{SD} 
= \frac{1}{\sum_{t\in\mathcal{F}} |A_t|} \sum_{t\in\mathcal{F}} |A_t|(\sigma_{t,1}^2 + \sigma_{t,2}^2 + \sigma_{t,1}^{-2} + \sigma_{t,2}^{-2}),
\end{align}
where $\mathcal{F}$ is the set of all triangles, $|A_t|$ is the area of triangle $t$ on the input surface, and $\sigma_{t,i}$ is the $i$-th singular value of the deformation gradient of triangle $t$.
%We picked this distortion measure among other choices since it efficiently balances between angle and area distortion and preserves local injectivity, but our method can be used with other metrics as well.
%\vova{should we that seam penalties and distortion measure choices are analogues to AutoCuts (but our objective is defined differently?)}
%Also note that our seam penalty and distortion measures are analogous to those used in AutoCuts~\cite{Poranne2017Autocuts}, although the way we combine these terms is different.  
%
%With the energies normalized, $L$ is invariant to coordinate scale and resolution for meshes with the same shape. %\danny{removed as this claim - for mesh resolution it is not true - certainly helps for scaling though.}


\subsection{Adding Global Bijectivity}
Most texture applications additionally require a guarantee of global bijectivity. Following Jiang et al.'s\ \shortcite{Jiang2017Simplicial} observation we realize this additional constraint on our mapping by triangulating the void regions of each iterations updated UV map and then augmenting our distortion energy $E_d$ with an additional term, \emph{not included in the distortion bound constraint}, to form a collapse preventing energy for the added negative-space triangles during each optimization iteration. For details see Section\ \ref{sec:bijectivity}. 
%As these constraints are implicitly encoded in our constraint. The additional objective terms are not added as a term to the distortion bound. Moreover, our framework stays the same when solving this extended problem.

% \danny{Moved and reorganized a good portion of what was here to the next section.}



%%%%%%%%%%%%%%%%%%%%%%%%%%%%%%%%%%%%%%%%%%%%%%

%\paragraph{Joint Discrete-Continuous Search}
%
%Instead of trying to solve an ill-conditioned real-valued optimization by approximating $T$ with $U$ \cite{Poranne2017Autocuts}, we alternatingly optimize $T$ (in topology steps) and $U$ (in descent steps), augmenting the continuous search with a discrete topology search (Section~\ref{sec:DCSearch}).
%
%Descent steps deal with the continuous search
%$\min_U E_d$
%where $T$ is fixed and Newton-type method is applied (Section~\ref{sec:descentStep}).
%Topology steps deal with the discrete search (Section~\ref{sec:topologyStep}) where a neighboring topology is searched based on the first-order reduction it would cause in $L$.
%While each descent step is one complete Newton-type iteration with back-tracking line search, we use a sequence of topology steps alternated with descent steps to perform a forward line search in topology space: a search direction is first decided and then the step size is increased to ensure sufficent energy decrease.
%
%Given a topology change, we evaluate the first-order reduction of $L$ it would cause by minimizing $E_d$ only on the local stencil of the edited seam edge and combine it with the $E_s$ change. Experiments demonstrate that this is efficient and sufficient for always editing the seams reasonably and sustainably (Section~\ref{sec:results_exp}).
%
%Since we ensure $L$ to be monotonically decreased in both the descent steps and topology steps, we can easily prove that a near-stationary point can be reached for any input within a bounded number of alternations (Section~\ref{sec:convergence}).

%\paragraph{Self-Weighted Objective}
%
%To automatically decide $\lambda$ and enable users to directly control the distortion they expect for the output UV map, we formulate a constrained optimization seeking to minimize seam energy $E_s$ subject to user-specified distortion bounds $b_d$:
%\[ \min_{T,U} E_s(T) \quad s.t. \quad E_d(T,U) - b_d \leq 0 \]
%Introducing $\lambda$ as a dual variable gives us
%\[ \min_{T,U} \max_{\lambda \in \mathbb{R^+}} E_s(T) + \lambda(E_d(T,U) - b_d) \]
%We solve it by adding a regularization term to make it smooth for $\lambda$, and alternatingly updating dual variable $\lambda$ and primal variables $T$ and $U$ (Section~\ref{sec:self_weighting}).
%
%The challenge here is that the initial embedding we start from is usually with high distortion, thus infeasible, and the quality of the output seams is highly affected by the $\lambda$ updating scheme. \minchen{[TODO] briefly describe what we do after finalized}
 
 %\danny{I haven't commented it the next paragraph but would suggest we: 1. remove this next paragraph from paper, 2. make bijection a main part of the algorithm and 3. move everything else to future work (if we don't do it) or results (if we do get to it).}\justin{cool w me} 

% \vova{
% I think the structure is a bit confusing (e.g., problem is not entirely defined in ``problem statement'', since we lack precise energy formulation, self-weighted objective talks about outer-look  (alg 1) and then about self-weighting. 
% %
% I propose the following structure: \\
% Section 3: first three paragraphs followed by ``dual objective'' (i.e., exclude "concretely we formulate ...)\\
%  - then write ``energy definition'' paragraph (from sec 5.3 and ``concretely we formulate'' paragraph) \\
 % - then write ``energy with global bijectivity'' paragraph (start ``for most texture applications one needs to ensure global bijecivity, we incorporate this into our energy following Jiang2017...'') \\
 % - [this and below is optional] then write ``optimization framework'' paragraph with Algortihm 1 and first paragraph from section 4. Here reference section 4 for dual update and section 5 for primal update. \\
 % - Note due to the last paragraph, we might change this section from ``Problem Statement'' to ``Optimization Framework''. Alternatively, we can have a separate section (between current 3 and 4, titled ``optimization framework'')\\
 % - Initialization (obtaining $T^0, U^0$ is not described anywhere. We should at least mention it in ``optimization framework'' part of the text. \\
% Section 4: rename into Dual Update \\
% Section 5: rename into Primal Update
% }

%\vova{I suggest we consistently use paragraphs or subsections across sec 3, 4, 5}

% !TeX root = OptCuts.tex

\section{Optimization Framework}

To solve our constrained optimization (\ref{eq:p1}) we focus on the saddle-point problem (\ref{eq:p2}). To do so we alternate between improving our primal variables $(T, U)$ and dual variable $\lambda$ starting from an initial, valid UV map $(T^0, U^0)$ and dual iterate $\lambda^0 = 0$ (Algorithm~\ref{alg:selfWeight}).

\begin{algorithm}[!h]
\SetAlgoLined
\KwData{$M$, $T^0$, $U^0$, $b_d$}
\KwResult{$T^*$, $U^*$}

$\lambda^0 \leftarrow 0$, $k \leftarrow 1$\;

\Do{either primal or dual not converged}{
  $\lambda^{k}$ $\leftarrow$ dualUpdate($T^{k-1}$, $U^{k-1}$, $b_d$, $\lambda^{k-1}$); // Section~\ref{sec:dualUpdate}\\

  $(T^{k}, U^{k})$ $\leftarrow$ primalUpdate($M$, $T^{k-1}$, $U^{k-1}$, $\lambda^{k}$); // Section~\ref{sec:primalUpdate}\\

  $k \leftarrow k + 1$\;
}
$(T^*, U^*)$ $\leftarrow$ $(T^{k-1}, U^{k-1})$\; 

\caption{OptCuts}
\label{alg:selfWeight}
\end{algorithm}

%\danny{Suggest moving this to our implementation section 
%\subsection{Initialization}
%To obtain an initial UV map for an input surface, we map its initial seam to a circle preserving edge length and parameterize the rest of the vertices through Tutte embedding with uniform weights.
%
%We compute initial seams for different surfaces according to their topology and geometry. For disk-topology surfaces, we simply pick their longest boundary as the initial seam. For genus-0 closed surfaces, we randomly pick 2 connected edges as the initial seam. \minchen{[TODO] change to curvest one point cut or farthest point cut if they are better} For high-genus surfaces, we follow Crane et al.~\shortcite{Crane:2013:DGP} to detect homology generators and connect all of them as the initial seam \minchen{[TODO]}.
%
%We simply start by ignoring the distortion constraints with $\lambda$ set to $0$, and let our dual update to modify $\lambda$ according to the intermediate distortions.

\subsection{Dual Update}
\label{sec:self_weighting}
\label{sec:dualUpdate}

% Our overall minimization is inequality constrained with a specified upper bound $b \in \mathbb{R}_+$ on distortion. \justin{I moved a parenthetical to a Minchen comment assuming he'll write it more formally}\minchen{(L2 norm on SD  energy for now - pretty easy to modify to an extremal measure if we want later on.)}

%
In this section we describe the update for the dual variable $\lambda^k$ given the current UV map $(T^{k-1}, U^{k-1})$.

The inner objective\minchen{Danny: which?} in the saddle-point problem (Eq.~\ref{eq:p2}) is nonsmooth in $\lambda$ since it does not take into account the fact that we might start away from feasibility and want to iteratively improve both our primal variables $(T, U)$ and our dual variable $\lambda$. To smoothly update to a current $\lambda^{k}$ in iteration $k$ from a previous estimate $\lambda^{k-1}$, we add a simple quadratic regularizer $R(\lambda,\lambda^{k-1}) = \frac{1}{2\kappa} (\lambda- \lambda^{k-1})^2$ \justin{to what?} to make sure $\lambda$ iterates behave reasonably. \justin{Right now previous sentence sounds quite heuristic; can you cite an optimization algorithm that does this?} In practice, we simply set $\kappa = 1$.

For iteration $k$ this gives us 
\[ \min_{T,V} \max_{\lambda \geq 0} E_{s}(T^{k-1}) + \lambda \big( E_{d}(T^{k-1}, U^{k-1}) - b_d\big) - \frac{1}{2\kappa} (\lambda- \lambda^{k-1})^2 \]
which can be solved in closed form as
\[ \lambda^{k} \leftarrow \max\big(0,\kappa \big( E_{d}(T^{k-1}, U^{k-1}) -b \big) + \lambda^{k-1}\big) \]

\danny{(Notice that throughout the above we can define a progressive $\lambda$ without needing to employ subgradients to reason about nonsmoothness in our sparsity energy.)}\justin{didn't follow this, not sure it's needed}

\subsection{Primal Update}
\label{sec:primalUpdate}

Our $(k+1)$-st primal update is a joint discrete-continuous search procedure that minimizes \eqref{eq:L} for a fixed $\lambda^{k+1}$ starting from $(T^k, U^k)$.
Rather than trying to approximate $T$ with non-smooth energies of $U$ and solve a potentially ill-conditioned real-valued optimization, we alternate local optimizations over $U$ and then $T$ in each inner iteration and obtain $(T^{k+1}, U^{k+1})$ when both are converged (Algorithm~\ref{alg:DCSearch}).

\begin{algorithm}[h]
\SetAlgoLined
\KwData{$M$, $T^{k}$, $U^{k}$, $\lambda^{k+1}$}
\KwResult{$T^{k+1}$, $U^{k+1}$}
$i \leftarrow 1$, $T^{k,0} \leftarrow T^{k}$, $U^{k,0} \leftarrow U^{k}$\;
$\delta^{k,0} \leftarrow 0$\;
\Do{either step is not converged}
{
	($T^{k,i}$, $U_a^{k,i-1}$) $\leftarrow$ topologyDescentStep($M$, $T^{k,i-1}$, $U^{k,i-1}$, $\delta^{k,i-1}$, $\lambda^{k+1}$); // Section~\ref{sec:topologySearch}\\
	($U^{k,i}$, $\delta^{k,i}$) $\leftarrow$ smoothDescentStep($M$, $T^{k,i}$, $U_a^{k,i-1}$); // Section~\ref{sec:descentStep}\\
	$i \leftarrow i+1$\;
}
$T^{k+1} \leftarrow T^{k,i-1}$, $U^{k+1} \leftarrow U^{k,i-1}$
\caption{Primal Update $k+1$}
\label{alg:DCSearch}
\end{algorithm}

The discrete topology search is performed in each topology descent step by querying a set of neighboring topologies and changing to the one with largest first-order reduction in $L$ if this reduction is prominent (Section~\ref{sec:topologySearch}). 
%
The continuous search is peformed in each smooth descent step, where we conduct a complete Newton-type iteration with backtracking line search towards minimizing distortion $E_d$ over vertices $U$ while holding topology, $T$, fixed (Section~\ref{sec:descentStep}).

By thresholding topology changes using the energy decrease of the last smooth descent step, we always proceed either the discrete topology search or the continuous search that decreases $L$ more.\justin{didn't follow previous sentence.}

% \vova{The section below seem to be concerned with convergence of Alg 1, which is beyond self-weighted objective, IMHO... 
% It is also confusing that there a subsection 5.7 on convergence...}
\subsection{Convergence}

Our primal update converges when the UV map reaches the local minimizer of $L$ in both continuous and discrete searches. Similar to standard continuous search, our discrete topology search also converges at a local minimum, but in topology space, where changing to all neighboring topologies can not decrease $L$. Since we ensured monotonic decrease in $L$ over both the smooth and topology descent steps, we can prove that a near-stationary point with respect to both $T$ and $U$ can be reached for any input within a bounded number of alternations given a fixed lambda (Section~\ref{sec:convergence}).

Theoretically, the dual update converges in two situations: either at $\lambda^* = 0$ when the primal minimizer $(T^*, U^*)$ is inside the feasible region ($E_d(T^*, U^*) < b_d$) with no seams, or at some $\lambda^* > 0$ with $(T^*, U^*)$ on the boundary of the feasible region ($E_d(T^*, U^*) = b_d$). However, for the latter case, since there is only a finite number of configurations of $T$, the near stationary $E_d$'s may not be exactly equal to $b_d$. Instead of setting a relatively large tolerance for detecting convergence, we stop when it first tries to violate distortion bound $b_d$ from being feasible. \justin{this was kind of vague; is it in the pseudocode?}


\section{Descent Steps for Continuous Optimization}

\subsection{Newton-type Iterations}

for each descent step inner iteration $j$:

compute $E_{SD}$ Hessian proxy $P^j$ using projected Newton;

compute $E_{SD}$ gradient $g^j$;

solve for search direction $p^j$ ($P^j p^j = -g^j$) using PARDISO symmetric indefinite solver;

compute initial step size $\alpha^j_0$ by avoiding element inversion;

backtracking line search with Armijo rule;

update $U^{j+1} = U^j + \alpha^j p^j$;

record energy decrease $(1-\lambda_t)\Delta E_{SD}^j$;

\subsection{Potential Accelerations for Practical Use}

Since our topological operations only change the mesh locally both on connectivity and coordinates, we could also update the Hessian or the decomposition locally after topology changes to save time. Besides, it's also interesting to try other Hessian approximation methods like L-BFGS or Majorization to explore further acceleration by finding a balance between computational cost and convergence rate.

For convergence tolerance of descent steps, $||\nabla E_{SD}||^2 \leq 10^{-6}$ (note that our energy is normalized) works generally well for all input models judging from the initiated fracture in the following topology step. In fact more inexact solve performs well on most of the models with even $||\nabla E_{SD}||^2 \leq 10^{-4}$, but some may result even better with $||\nabla E_{SD}||^2 \leq 10^{-8}$. Since we are conducting non-convex optimization, $||\nabla E_{SD}||^2$ is not always decreasing, which is also why we don't use Wolfe conditions for line search. The argument here for tolerance issue is that, it depends on whether we are truly in the infinitesimal region of a stationary. Some configuration with $||\nabla E_{SD}||^2 \leq 10^{-6}$ may still not inside the infinitesimal region of a stationary, where if optimization goes on, the $||\nabla E_{SD}||^2$ will go up and then fall down again to a real stationary, which is understandable in non-convex optimization.

\section{Topology Steps for Discrete Optimization}

\subsection{Evaluating Topological Operations via Optimization on Local Stencils}

Candidate Filtering:
for each vertex
  compute divergence of local gradients
independently picking $\sqrt{n_{v,b}^i}$ boundary vertices and $\sqrt{n_{v,i}^i}$ interior vertices with largest divergence as candidates

Local Evaluation:
for each candidate vertex
  if on boundary
    for each interior incident edge
      split and compute $\Delta E_{SD,l}$ locally
      compute $\Delta E_{w,l} = (1 - \lambda_t) \Delta E_{SD,l} + \lambda_t \Delta E_{se}$
  else
    for each pair of incident edges forming a smooth path
      split and compute $\Delta E_{SD,l}$ locally
      compute $\Delta E_{w,l} = 0.5((1 - \lambda_t) \Delta E_{SD,l} + \lambda_t \Delta E_{se})$

split the vertex with largest $|\Delta E_{w,l}|$
turn on fracture propagation

\textcolor{red}{try larger stencils}

\textcolor{red}{enable merge operation}

\textcolor{red}{\subsection{Line Search in Topological Space}}

Current Fracture Propagation:
if fracture propagation is on
  for each fracture tail vertex $k$
    for each interior incident edge of $k$
      split and compute $\Delta E_{SD, l}$ locally
      compute $\Delta E_{w,l} = (1 - \lambda_t) \Delta E_{SD,l} + \lambda_t \Delta E_{se}$
  if the largest $|\Delta E_{w,l}|$ is larger than $|(1-\lambda_t)\Delta E_{SD}^j|$
    propagate fracture by splitting the vertex
  else
    turn off fracture propagation for the rest of the current descent step

% !TeX root = OptCuts.tex

\section{Results and Discussion}
\label{sec:results}

\subsection{Experiments}

\paragraph{Initial Embedding} For disk-topology surfaces, we map the longest boundary to a circle preserving edge length and obtain a Tutte embedding with uniform weights \minchen{[TODO] see whether using MVC weights converge faster}. For closed surfaces (including high-genus surfaces), we first apply some simple heuristics to obtain an initial seam, and then treat them as disk-topology surfaces. 
The heuristics for genus-0 surfaces include farthest point cut and random/curvest one point cut \minchen{[TODO] decide one and add description, or use all?}; for high-genus surfaces, we follow Crane et al.~\shortcite{Crane:2013:DGP} to detect homology generators and then cut along all of them \minchen{[TODO]}.
In order to show that we search for locally optimal UV maps regardless of the given initial embedding, we run our method starting from triangle soup and preliminary UV maps produced by other methods or by the users. The output maps are still with high quality (Figure~\ref{fig:bad_init_still_ends_well}) \minchen{[TODO]}.

\paragraph{Quality and Timing Comparisons} \minchen{[TODO], also provide detailed settings on the compared methods, and how much user assistance was needed for other methods?} We demonstrate our framework's capabilities by first comparing to AutoCuts~\cite{Poranne2017Autocuts} and two typical classic seam cutting methods~\cite{Gu2002Geometry,Sheffer2002Seamster}. Given the same input surface and initial UV map, we efficiently reach identical distortion bounds with shorter seam lengths (Figure~\ref{fig:QT_comp}). 
When we change the settings in order to obtain nearly isometric UV maps, the quality of the seams by other methods drop drastically while our method keeps generating high-quality seams (Figure~\ref{fig:strict_bounds_comp}).

\paragraph{Triangulation Invariance} \minchen{[TODO] not sure whether applicable}

\paragraph{Scalability Test} \minchen{[TODO]}

\subsection{Variations}

Without changing the framework, simply reformulating $E_w = E_s + \lambda E_d$ according to different needs enables OptCuts to solve mesh parameterization problems in many variations:

\paragraph{Global Bijectivity} \minchen{[DOING]} Augmenting our $E_s$ with a collision handling energy $E_b$ will easily achieve joint seam placement and bijective mesh parameterization. We show that by adding a scaffold mesh~\cite{Jiang2017Simplicial} to the voided regions of the UV map and preventing the scaffold mesh from degenerate, our method automatically generate high-quality bijective maps with optimal seams different from that of locally injective parameterization (Figure~\ref{fig:bijective_vs_injective}).

\paragraph{Conformal Parameterization} \minchen{[TODO]} Using a conformal energy~\cite{Hormann2000MIPS,Sheffer2005ABFPP} for $E_s$ will achieve joint seam placement and conformal parameterization. Figure~\ref{fig:conformal_vs_isometry} shows some results with $E_s = E_{ABForMIPS}$~\cite{} compared to results with $E_s = E_{SD}$, where different seams are generated while our framework stays the same.

\paragraph{Regional Seam Placement} \minchen{[TODO]} On the discrete side, if we reweight $E_{SL}$ with an edge prior provided by the user or an algorithm~\cite{} as
\[ E_s = \hat{E}_{SL} = \sum_{i\in\mathcal{S}} w_{SL,i} E_{SL,i} \quad w_{SL,i} \in \mathcal{R^+} \]
we could bias the seam placement towards regions e.g. where continuity is less in demand (Figure~\ref{fig:regional_seam_placement}).

discoveries? like interior splits?

\section{Conclusions and Future Works}

\begin{itemize}
\item Our method doesn't provide globally optimal solutions, the results are still locally optimal, but w.r.t. both seams and distortion, which is better than previous 2-pass methods that breaks the correlation between seams and distortion.
\item take advantage of basic SIMD type of parallelism for accelerating query and improving results' quality by directly evaluating $f_v$ for neighbors and track multiple branches, very useful for practical implementations
\item if the user won't mind getting a slightly different triangulation, we could also create fractures in the interior of an element and locally remesh the stencil
\item start and solve in 3D by reducing curvature so that the need for locally injective initial embedding in parameterization problems could be eliminated, and the result is only "biased" by it's 3D shape, which is the most reasonable bias
\item try conformal energy like MIPS
\item \textcolor{red}{bijectivity}, seamless, and other augmentation of continuous energy?
\item handle user preferences on seam placement
\item seam smoothness, patch related discrete energy augmentation?
\end{itemize}

\section{Acknowledgements}

\bibliographystyle{ACM-Reference-Format}
\bibliography{OptCuts} 

\end{document}
